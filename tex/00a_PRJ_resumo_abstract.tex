%%________________________________________________________________________
%% LEIM | PROJETO
%% 2022 / 2013 / 2012
%% Modelo para relatório
%% v04: alteração ADEETC para DEETC; outros ajustes
%% v03: correção de gralhas
%% v02: inclui anexo sobre utilização sistema controlo de versões
%% v01: original
%% PTS / MAR.2022 / MAI.2013 / 23.MAI.2012 (construído)
%%________________________________________________________________________




%%________________________________________________________________________
\myPrefaceChapter{Resumo}
%%________________________________________________________________________

A análise de dados é fundamental para a tomada de decisões estratégicas no simulador de negócios Marketplace Simulations - International Corporate Management, utilizado no ISCAL. No entanto, a plataforma original apresenta limitações na visualização e filtragem de dados, dificultando a análise eficiente por parte dos estudantes.

Este projeto desenvolve uma aplicação web que resolve estas limitações, permitindo que grupos de estudantes façam o carregamento de ficheiros XLSX exportados do simulador, organizados por trimestres, e visualizem os dados através de gráficos interativos. A aplicação oferece funcionalidades avançadas de filtragem que não estão disponíveis na plataforma original, facilitando a análise de dados e a tomada de decisões estratégicas.

A solução foi implementada em Django com Flowbite para a interface, utilizando Pandas para processamento de dados e Plotly para visualização. A plataforma garante o isolamento de dados entre grupos e uma experiência de utilizador intuitiva, permitindo uma análise de dados mais eficiente e estruturada.






%%________________________________________________________________________
\myPrefaceChapter{Abstract}
%%________________________________________________________________________

Data analysis is crucial for strategic decision-making in the Marketplace Simulations - International Corporate Management business simulator, used at ISCAL. However, the original platform has limitations in data visualization and filtering, making efficient analysis challenging for students.

This project develops a web application that addresses these limitations, allowing student groups to upload XLSX files exported from the simulator, organized by quarters, and view the data through interactive charts. The application provides advanced filtering capabilities not available in the original platform, facilitating data analysis and strategic decision-making.

The solution was implemented in Django with Flowbite for the interface, using Pandas for data processing and Plotly for visualization. The platform ensures data isolation between groups and an intuitive user experience, enabling more efficient and structured data analysis.
