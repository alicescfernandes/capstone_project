%%________________________________________________________________________
%% LEIM | PROJETO
%% 2022 / 2013 / 2012
%% Modelo para relatório
%% v04: alteração ADEETC para DEETC; outros ajustes
%% v03: correção de gralhas
%% v02: inclui anexo sobre utilização sistema controlo de versões
%% v01: original
%% PTS / MAR.2022 / MAI.2013 / 23.MAI.2012 (construído)
%%________________________________________________________________________
%%
%%
%% É importante ver o essencial do LaTeX antes de usar este template.
%% um bom "ponto de partida": http://en.wikibooks.org/wiki/LaTeX/
%%
%%________________________________________________________________________
%% Para alterar o Título, Nome dos Autores e Nome dos Orientadores fazer:
%% - abrir o ficheiro "00_PRJ_padrao.sty"
%% - procurar "Nome do aluno" e alterar
%% - procurar "número" e alterar; deixar os parêntesis
%% - procurar "Nome do orientador" e alterar
%% - procurar "[Doutor]" e atirar os parêntesis rectos ou eliminar tudo
%%
%% Caso precise de adicionar um novo aluno, ou orientador, deve:
%% - selecionar toda a linha (com "Nome do aluno" ou "Nome do orientador")
%% - seleccionar a linha imediatamente acima dessa
%% - copiar ambas as linhas
%% - colocar as linhas copiadas logo abaixo da primeira linha seleccionada
%%________________________________________________________________________





%% o documento é definido do tipo "book" em a4, font 12pt
\documentclass[a4paper,12pt]{book}

\usepackage[acronym, nomain]{glossaries}
\makeglossaries

\usepackage{etoolbox}

\newcommand{\acfoot}[1]{%
  \acrshort{#1}\footnote{\acrlong{#1}}%
}

\usepackage{float}

\usepackage[export]{adjustbox}% 'export' allows adjustbox keys in \includegraphics

%% para usar caracteres acentuados
\usepackage[utf8]{inputenc} % no Unix (com codificação UTF-8)
%% \usepackage[latin1]{inputenc} % no Windows (com codificação ISO-8859-1, ou ISO-8859-15)

% para aspectos de hifenização (do Português)
\usepackage[portuguese]{babel}

%% para incluir imagens e tratar de modo adequado endereços url
\usepackage{graphicx,url}

%% para usar \begin{comment} ... \end{comment}
\usepackage{verbatim}

%% para usar o símbolo do euro
%usepackage{eurosym}

%% para juntar multiplas linhas em tabelas
\usepackage{multirow}

%% para usar símbolos matemáticos
\usepackage[centertags]{amsmath}

%% para usar \mathds{...}
\usepackage{dsfont}

%% para formatação (e.g. espaçamento) de Tabelas
\usepackage{tabls}

%% para usar listagens de código
\usepackage{listings}
\usepackage{xcolor} % Para cores nos blocos, opcional mas útil
%% para escrita "formal" de algoritmos
\usepackage{algorithm}

%% Para que a numeracao de listings seja global
%% (e não no contexto de cada capitulo!)
\usepackage{chngcntr}

% para usar um estilo diferente na identificação de "Capítulo"
%Sonny, Lenny(x), Glenn, Conny, Rejne, Bjarne
%\usepackage[Lenny]{fncychap}
% Para ter no "heading" o nome do capítulo e da secção
\usepackage{fancyhdr}



% para usar hiperligações (especialmente relevante para o índice)
\usepackage[hidelinks]{hyperref}
%\hypersetup{linktocpage}
%\hypersetup{
%    colorlinks,
%    citecolor=black,
%    filecolor=black,
%    linkcolor=black,
%    urlcolor=black
%}

%%________________________________________________________________________
%% para incluir o estilo proposto
\usepackage{./00_PRJ_padrao}
%%________________________________________________________________________



%%________________________________________________________________________
% mudar alguns nomes fixos para Português
%%________________________________________________________________________
% renomear "Listing" para "Código"
\renewcommand{\lstlistingname}{Código}
\addto\captionsportuges{
    \renewcommand{\contentsname}{Apêndice}}
\addto\captionsportuges{
    \renewcommand{\bibname}{Bibliografia}}
\addto\captionsportuges{
    \renewcommand{\proofname}{Prova}}
\addto\captionsportuges{
    \renewcommand{\chaptername}{Capítulo}}
\setlength{\headheight}{15pt}

\newcommand{\myListOfAcronymns}{
    \cleardoublepage
    \phantomsection
    \addcontentsline{toc}{chapter}{Lista de Acrónimos}
    \printacronyms[type=\acronymtype, title=Lista de Acrónimos]
}
%%________________________________________________________________________


%%________________________________________________________________________
%%%%%%%%%%%%%%%%%%%%%%%%%%%%%%%%%%%%%%%%%%%%%%%%%%%%%%%%%%%%%%%%%%%%%%%%%%
%% Begin Document
%%%%%%%%%%%%%%%%%%%%%%%%%%%%%%%%%%%%%%%%%%%%%%%%%%%%%%%%%%%%%%%%%%%%%%%%%%

\lstset{
  basicstyle=\ttfamily\small,
  breaklines=true,
  breakatwhitespace=false,
  frame=single,
  numbers=left,
  numberstyle=\tiny,
  stepnumber=1,
  numbersep=5pt,
  tabsize=2,
  showstringspaces=false,
  columns=fullflexible,
  keepspaces=true,
  captionpos=b
}

\setlength{\parskip}{0.5em}

\usepackage{titlesec}

\titlespacing*{\subsubsection}
  {0pt}{10pt plus 2pt minus 2pt}{5pt plus 2pt minus 2pt}

\begin{document}

\newacronym{vscode}{VSCode}{\textit{Visual Studio Code}}
\newacronym{csv}{CSV}{\textit{Comma-Separated Values}}
\newacronym{ui}{UI}{\textit{User Interface}}
\newacronym{ux}{UX}{\textit{User Experience}}
\newacronym{api}{API}{\textit{Application Programming Interface}}
\newacronym{xlsx}{XLSX}{\textit{Excel Spreadsheet Format}}
\newacronym{iscal}{ISCAL}{Instituto Superior de Contabilidade e Administração de Lisboa}
\newacronym{isel}{ISEL}{Instituto Superior de Engenharia de Lisboa}
\newacronym{captcha}{CAPTCHA}{\textit{Completely Automated Public Turing test to tell Computers and Humans Apart}}
\newacronym{sad}{SAD}{Sistemas de Apoio à Decisão}
\newacronym{uml}{UML}{\textit{Unified Modeling Language}}
\newacronym{etl}{ETL}{\textit{Extract, Transform, Load}}
\newacronym{mvc}{MVC}{\textit{Model-View-Controller}}
\newacronym{orm}{ORM}{\textit{Object-Relational Mapping}}
\newacronym{sql}{SQL}{\textit{Structured Query Language}}
\newacronym{crm}{CRM}{\textit{Customer Relationship Management}}
\newacronym{cms}{CMS}{\textit{Content Management System}}
\newacronym{json}{JSON}{\textit{JavaScript Object Notation}}
\newacronym{html}{HTML}{\textit{HyperText Markup Language}}
\newacronym{css}{CSS}{\textit{Cascading Style Sheets}}
\newacronym{js}{JS}{\textit{JavaScript}}
\newacronym{dom}{DOM}{\textit{Document Object Model}}
\newacronym{spa}{SPA}{\textit{Single Page Application}}
\newacronym{wsgi}{WSGI}{\textit{Web Server Gateway Interface}}
\newacronym{uuid}{UUID}{\textit{Universally Unique Identifier}}
\newacronym{uc}{UC}{\textit{Use Case}}
\newacronym{http}{HTTP}{\textit{Hypertext Transfer Protocol}}
\newacronym{wcag}{WCAG}{\textit{Web Content Accessibility Guidelines}}
\newacronym{cli}{CLI}{\textit{Command Line Interface}}
\newacronym{seo}{SEO}{\textit{Search Engine Optimization}}
\newacronym{gzip}{GZIP}{\textit{GNU Zip}}
\newacronym{aria}{ARIA}{\textit{Accessible Rich Internet Applications}}
\newacronym{rest}{REST}{\textit{Representational State Transfer}}
\newacronym{ssh}{SSH}{\textit{Secure Shell}}
\newacronym{vps}{VPS}{\textit{Virtual Private Server}}
\newacronym{mtv}{MTV}{\textit{Model-Template-View}}
\newacronym{url}{URL}{\textit{Uniform Resource Locator}}


% para que a numeracao de listings seja global
% (e não no contexto de cada capitulo!)
\counterwithout{lstlisting}{chapter}
%% incluir a capa
\frontmatter
\fazerCapa
%% incluir o resumo e abstract
%% caso pretenda, incluir os agradecimentos e a dedicatória
%%________________________________________________________________________
%% comentar o que não interessar
%%________________________________________________________________________
%% LEIM | PROJETO
%% 2022 / 2013 / 2012
%% Modelo para relatório
%% v04: alteração ADEETC para DEETC; outros ajustes
%% v03: correção de gralhas
%% v02: inclui anexo sobre utilização sistema controlo de versões
%% v01: original
%% PTS / MAR.2022 / MAI.2013 / 23.MAI.2012 (construído)
%%________________________________________________________________________




%%________________________________________________________________________
\myPrefaceChapter{Resumo}
%%________________________________________________________________________

Neste trabalho apresentamos o desenvolvimento de uma aplicação \textit{web} para visualização e análise de dados, criada com o objetivo de apoiar os alunos na interpretação da informação extraída da plataforma \textit{International Corporate Management}, utilizada na unidade curricular Projeto de Simulação em Negócios Internacionais da Licenciatura em Comércio e Negócios Internacionais do \gls{iscal}.

A plataforma permite extrair a informação apresentada aos alunos ao longo da simulação, organizada por diferentes áreas funcionais. No entanto, os dados extraídos apresentam uma grande diversidade estrutural, com problemas como nomes de colunas com quebras de linha ou com muitos espaços, presença de colunas irrelevantes e valores mal formatados. Como estes dados servem de base para visualizações e análises interativas, tornou-se necessário garantir que estivessem num formato tratado, coerente e estruturado. Para isso, foi desenvolvido uma \textit{pipeline} de normalização recorrendo a \textit{Python}, com operadores responsáveis por tarefas como a extração e normalização dos cabeçalhos, remoção de colunas desnecessárias, limpeza de conteúdo das células e transformação de valores nomeadamente no caso de métricas financeiras. Os nomes dos ficheiros normalizados são também criados com base na folha de origem, garantindo consistência. Este processo assegura que a informação fica preparada para ser visualizada forma coerente ao longo dos diferentes períodos da simulação.

A aplicação foi construída sobre o framework \textit{Django}, que assegura a gestão de contas de utilizador, a organização dos dados por períodos (quarters) e a separação da informação por utilizador. Esta arquitetura garante que cada conta tem acesso apenas aos seus próprios dados, promovendo o isolamento e a integridade da informação. A interface gráfica combina \textit{Flowbite}, \textit{Plotly} e \textit{WebComponents}, permitindo uma experiência de utilização fluida, intuitiva e compatível com diferentes dispositivos.

Do ponto de vista funcional, a aplicação permite criar e gerir períodos de simulação, carregar conjuntos de dados, extrair automaticamente a informação relevante, aplicar filtros por parâmetros como marca, cidade ou necessidade do cliente, e visualizar gráficos configurados com base na estrutura dos dados reconhecidos.

A solução foi desenhada com foco na modularidade do \textit{pipeline} de transformação e na experiência de utilização, com o objetivo de apoiar a tomada de decisão por parte dos alunos durante a simulação.

%%________________________________________________________________________
\myPrefaceChapter{Abstract}
%%________________________________________________________________________

In this paper, we present the development of a web application for data visualization and analysis, created with the aim of supporting students in interpreting information extracted from the International Corporate Management platform, used in the International Business Simulation Project course of the Bachelor's Degree in International Trade and Business at \gls{iscal}.

The platform allows the extraction of information presented to students throughout the simulation, organized by different functional areas. However, the extracted data presents a great structural diversity, with problems such as column names with line breaks or too many spaces, the presence of irrelevant columns, and poorly formatted values. As this data serves as the basis for interactive visualizations and analyses, it became necessary to ensure that it was in a treated, coherent, and structured format. To this end, a normalization pipeline was developed using Python, with operators responsible for tasks such as extracting and normalizing headers, removing unnecessary columns, cleaning cell content, and transforming values, particularly in the case of financial metrics. The names of the normalized files are also created based on the source sheet, ensuring consistency. This process ensures that the information is prepared to be viewed consistently throughout the different periods of the simulation.

The application was built on the Django framework, which ensures user account management, data organization by periods (quarters), and separation of information by user. This architecture ensures that each account has access only to its own data, promoting isolation and information integrity. The graphical interface combines Flowbite, Plotly, and WebComponents, allowing for a fluid, intuitive user experience that is compatible with different devices.

From a functional point of view, the application allows you to create and manage simulation periods, load data sets, automatically extract relevant information, apply filters by parameters such as brand, city, or customer need, and view graphs configured based on the structure of the recognized data.

The solution was designed with a focus on the modularity of the transformation pipeline and the user experience, with the aim of supporting students' decision-making during the simulation.

%%________________________________________________________________________
%% LEIM | PROJETO
%% 2022 / 2013 / 2012
%% Modelo para relatório
%% v04: alteração ADEETC para DEETC; outros ajustes
%% v03: correção de gralhas
%% v02: inclui anexo sobre utilização sistema controlo de versões
%% v01: original
%% PTS / MAR.2022 / MAI.2013 / 23.MAI.2012 (construído)
%%________________________________________________________________________




%%________________________________________________________________________
\myPrefaceChapter{Agradecimentos}
%%________________________________________________________________________

Ao longo do desenvolvimento deste projeto, tive a oportunidade de contar com a orientação de docentes do ISEL e do ISCAL, a quem agradeço pelo o apoio dado durante o projeto. 

Ao Professor \textbf{Paulo Trigo}, o Professor \textbf{Helder Fanha} e o \textbf{Professor 2 do ISEL}, que, em diferentes momentos, contribuíram com sugestões, comentários e sugestões. O seu acompanhamento ajudou a resolver dúvidas que surgiram dorante o desenvolvimento, e a validar algumas das opções técnicas.

Fica aqui, então, o meu agradecimento pelo contributo dado ao longo deste processo.

\include{./00c_PRJ_dedicatoria}
%%________________________________________________________________________

%% incluir a lista de conteúdos (índice, tabelas e figuras)
%comentar se quiser alterar o espaçamento entre linhas
\setlinespacing{1.15}
\myTableOfContents
%%________________________________________________________________________
%% comentar o que não interessar

\myListOfTables
\myListOfFigures
\myListOfAcronymns
%%________________________________________________________________________





%% "Corpo Principal" do texto
\mainmatter
\myPageStyle

%% cada um dos Capítulos
%% (aqui separados em dois ficheiros)
%%________________________________________________________________________
%% comentar o que não interessar (ou incluir outros ficheiros)
%%________________________________________________________________________
%% LEIM | PROJETO
%% 2022 / 2013 / 2012
%% Modelo para relatório
%% v04: alteração ADEETC para DEETC; outros ajustes
%% v03: correção de gralhas
%% v02: inclui anexo sobre utilização aplicação controlo de versões
%% v01: original
%% PTS / MAR.2022 / MAI.2013 / 23.MAI.2012 (construído)
%%________________________________________________________________________


%%________________________________________________________________________
\chapter{Introdução}
\label{ch:introducao}
%%________________________________________________________________________

No contexto do ensino superior, tem-se dado cada vez mais importância à integração de ferramentas tecnológicas que tornem o processo de aprendizagem mais prático, como é o caso do simulador utilizado pelos alunos do \gls{iscal}. Esta ferramenta permite que os estudantes apliquem, de forma interativa, os conhecimentos adquiridos ao longo do curso, colocando-os em situações de tomada de decisão semelhantes às que enfrentariam num ambiente real. Deste modo, o simulador não só reforça os conteúdos teóricos, como também reforça o desenvolvimento das competências aprendidas. 

Neste contexto, a análise eficiente de dados torna-se essencial para a tomada de decisões e tem impacto na avaliação final dos alunos, no entanto, a complexidade e a falta de uma ferramenta visual que ajude a perceber as informações apresentadas pode representar um desafio grande para os estudantes.

\section{Motivação}

O presente projeto surgiu da necessidade  entre os alunos do \gls{iscal} que utilizam o simulador \textit{International Corporate Management} da empresa \textit{Marketplace Simulations}, que é um simulador de negócios internacionais e que iremos descrever no capítulo~\ref{sec:marketplace}, onde os alunos agrupam-se em empresas e simulam um negócio num mercado internacional. Embora a plataforma apresente toda a informação necessária à tomada de decisões na simulação, esses dados estão espalhados em múltiplas páginas e apresentados em tabelas, com poucas funcionalidades de visualização. Esta limitação obriga os alunos a alternar entre páginas, copiar dados manualmente ou criar folhas de cálculo externas, comprometendo a eficiência e a análise da informação.

\section{Objetivos}

A aplicação proposta neste relatório pretende ajudar nesse sentido, disponibilizando uma interface que permite aos utilizadores carregar dados retirados da plataforma de simulação e tornar esses ficheiros em visualizações que podem ser consultadas e manipuladas. A aplicação permitirá aos utilizadores:

\begin{itemize}
    \item Criar uma conta na plataforma que suporte a persistência da informação carregada.
    \item Carregar ficheiros exportados da plataforma de simulação.
    \item Visualizar os dados em gráficos interativos.
    \item Conseguir ter uma experiência de utilização intuitiva e fácil.
\end{itemize}

Pretende-se também que a aplicação adote uma arquitetura simples de manter e que utilize os mesmos conceitos que a plataforma de simulação, dando importância aos seguintes itens:

\begin{itemize}
    \item Normalização e transformação automática de dados carregados.
    \item Facilidade na gestão de ficheiros, com o objetivo de oferecer uma interface intuitiva para os utilizadores finais.
    \item Organizar a informação por utilizador, garantindo que cada utilizador apenas consegue consultar a sua informação carregada.
    \item Adotar um modelo de funcionamento semelhante à plataforma de simulação, de modo a tornar a utilização mais fácil e garantir que a nossa aplicação tenha fronteiras claras de utilização.
\end{itemize}

Ao longo deste relatório, serão apresentadas as decisões tomadas, bem como os fundamentos que orientaram o desenvolvimento da aplicação proposta.

%%________________________________________________________________________
\chapter{Trabalho Relacionado}
\label{ch:trabalhoRelacionado}
%%________________________________________________________________________

O presente projeto insere-se num contexto mais geral de ferramentas pedagógicas e de \gls{sad}, ainda que neste caso concreto, em ambientes simulados no ensino superior. No âmbito do \gls{iscal}, a utilização da plataforma \textit{Marketplace Simulations} \cite{MarketplaceSim_2025} permite aos estudantes desenvolver competências práticas em ambientes virtuais de negócios, simulando o funcionamento de mercados reais. A necessidade de suporte digital à análise e simulação motivou o desenvolvimento de outras ferramentas auxiliares, com destaque para um projeto também realizado em parceria com o \gls{iscal}, focado em simulações parciais de modelos económicos.

Esse projeto, embora partilhe uma motivação semelhante, segue uma abordagem diferente. Em particular, a aplicação referida acima permite simular cenários específicos com base em \textit{inputs} manuais, o que pode ser útil para quando se procura prever resultados com foco muito concreto (por exemplo, simular o impacto de uma única variável nos resultados). A nossa plataforma foca-se numa outra vertente, em que pretende ser uma ferramenta para auxiliar o ensino de forma prática.

O projeto assume então o uso de dados extraídos diretamente da plataforma de simulação, e a valorização da experiência do utilizador na apresentação de dados. Não será uma plataforma de simulação, mas sim uma ferramenta para auxiliar o ensino de forma prática.

\noindent \textbf{Sistemas de apoio à decisão}
\label{sec:sad}

Um \gls{sad} é uma aplicação, ou um conjunto de aplicações, pensadas para ajudar utilizadores a tomar decisões mais informadas. Ao contrário de sistemas que tomam decisões de forma autónoma, um \gls{sad} funciona como uma ferramenta que mostra aos utilizadores os dados e análises para que estes possam avaliar alternativas, prever resultados e escolher a melhor ação possível.

Na prática, um \gls{sad} é responsável por recolher, organizar e processar grandes volumes de dados, muitas vezes com origem em várias fontes, e transformá-los em informação relevante. Esta informação é normalmente apresentada através de tabelas, indicadores ou representações visuais como gráficos de barras. Um aspeto importante é a capacidade de aplicar filtros e explorar diferentes cenários, o que permite ao utilizador testar hipóteses, identificar padrões e comparar dados.

Estes sistemas são particularmente úteis em contextos onde há grande complexidade como por exemplo, na gestão empresarial, em análises financeiras, ou na análise de mercados. Num ambiente académico, como é o caso da plataforma usada neste projeto, um \gls{sad} ajuda os estudantes a perceber que decisões tomar, mostrando dados históricos para que possam ser identificadas tendências ou padrões.

É importante reforçar que o \gls{sad} não substitui o utilizador. Em vez disso, aumenta a capacidade do utilizador de interpretar a informação disponível e tomar decisões, reduzindo o risco de erro e aumentando a confiança. Estes sistemas podem ser plataformas criadas especificamente para o efeito, ou podem ser derivadas de outras plataformas como \textit{PowerBI}, \textit{Tableau}, \textit{Grafana}, ou outras ferramentas de visualização de dados.

%%________________________________________________________________________
\chapter{Modelo Proposto}
\label{ch:modeloProposto}
%%________________________________________________________________________

Neste capítulo vamos descrever o modelo que serviu de base à implementação da aplicação, com foco na forma como os dados são organizados, processados e apresentados ao utilizador. A estrutura proposta resulta das necessidades identificadas durante a análise dos ficheiros obtidos da plataforma de simulação, bem como dos requisitos definidos com os orientadores do projeto.

O modelo tem como objetivo garantir que a aplicação desenvolvida corresponde as necessidades identificadas. Para isso, foram definidos vários requisitos, e foi estudado como iriamos apresentar os dados tendo em conta os ficheiros recebidos, entre outras decisões que foram tomadas durante o desenvolvimento do projeto.

Nos próximos parágrafos vamos então detalhar cada uma das partes do modelo proposto, começando pelos requisitos funcionais e não funcionais que foram definidos.

%%________________________________________________________________________
\section{Requisitos}
\label{sec:requisitos}
%%________________________________________________________________________

\subsubsection{Requisitos funcionais}

Os requisitos funcionais descrevem as funcionalidades obrigatórias que a aplicação deve oferecer aos utilizadores. No contexto deste projeto, definem as ações que a aplicação deve ser capaz de executar.

Para o projeto, identificamos os seguintes requisitos, ordenados por prioridade. Os tabelas de requisitos por inteiro, incluindo algumas definições adicionais, estão incluídas em apêndice \ref{ch:tabRequisitos}.

\begin{itemize}
    \item \textbf{Visualização de dados:} O utilizador deve poder visualizar gráficos interativos baseados nos dados carregados, com a possibilidade de aplicar filtros como país ou \textit{quarter} selecionado.

    \item \textbf{Gestão de ficheiros:} O utilizador deve poder carregar ficheiros, associá-los a \textit{quarters} e eliminá-los quando necessário. A aplicação valida os formatos e garante que apenas ficheiros válidos são tratados.
    
    \item \textbf{Gestão de \textit{quarters}:} O utilizador deve poder criar períodos, identificados como \textit{Quarter N}, para organizar os ficheiros carregados. Tem também de ser possível visualizar a lista de \textit{quarters} disponíveis.

    \item \textbf{Autenticação:} O utilizador deve poder criar uma conta e usar autenticar com a conta criada. Os dados apresentados devem ser apenas os do utilizador autenticado.

    \item \textbf{Normalização de dados:} Os dados extraídos dos ficheiros devem ser automaticamente normalizados para garantir consistência e compatibilidade com a aplicação.
    
\end{itemize}

\subsubsection{Requisitos não funcionais}

Alguns requisitos não funcionais também foram importantes para garantir a estabilidade e a capacidade de interação da aplicação. Os requisitos identificados foram os seguintes:

\begin{itemize}
    \item \textbf{Usabilidade:} A aplicação deve ser intuitiva, suportar múltiplos \textit{browsers} e permitir o carregamento progressivo de gráficos sem bloquear a interface visual (\textit{lazy load}).
    
    \item \textbf{Performance:} A aplicação deve ser capaz de suportar múltiplos utilizadores em simultâneo, mantendo uma resposta rápida às interações.
    
    \item \textbf{Acessibilidade:} A aplicação deve ser acessível por teclado e compatível com leitores de ecrã (\textit{screen-reader friendly}).

\end{itemize}

\section{Casos de Utilização}
\label{ch:casosUtilizacao}

Com base nos requisitos funcionais e não funcionais identificados na secção~\ref{sec:requisitos}, foram definidos os casos de utilização que iremos suportar. A figura~\ref{fig:umlCasosUtilizacao} apresenta o diagrama \gls{uml} dos casos de utilização da aplicação. Este diagrama ilustra as interações entre os vários atores que fazem parte da aplicação.

\begin{figure}[h]
\centering
\includegraphics[max width=\textwidth]{./img/usecase_uml}
\caption{\gls{uml} dos Casos de Utilização}
\label{fig:umlCasosUtilizacao}
\end{figure}

Na figura~\ref{fig:umlCasosUtilizacao} podemos ver os atores e os casos de utilização da aplicação, que são os seguintes:
\begin{itemize}
    \item \textbf{Utilizador} É o ator principal, responsável por interagir com a aplicação. Pode criar conta, iniciar sessão, carregar ficheiros, visualizar gráficos, entre outras ações.
    \item \textbf{Base de Dados (DB)} Responsável por armazenar dados dos ficheiros carregados e metadados associados, e os \textit{quarters} criados pelo utilizador.
    \item \textbf{Sistema de Ficheiros} Responsável pelo armazenamento físico dos ficheiros carregados.
\end{itemize}


Temos também definidos os seguintes casos de utilização, que descrevemos com algum detalhe:
\begin{itemize}
    \item \textbf{Criar e gerir \textit{quarters}:} Permite ao utilizador, após autenticação, criar \textit{quarters}. Cada \textit{quarter} é identificado por um número (único por utilizador) e guardado na base de dados, sendo associado a um \gls{uuid} para efeitos internos.

    \item \textbf{Carregar ficheiros:} O utilizador seleciona um \textit{quarter} e carrega ficheiros exportados da plataforma de simulação. A aplicação valida os ficheiros, extrai e processa cada folha de cálculo e aplica a \textit{pipeline} de normalização. Os dados normalizados são armazenados e associados ao \textit{quarter} correspondente.

    \item \textbf{Eliminar ficheiros:} Permite ao utilizador remover ficheiros que foram carregados. A aplicação atualiza os registos no \textit{backend}, marca os dados antigos como não correntes e evita que sejam considerados nos gráficos, garantindo que apenas uma versão está ativa por ficheiro.

    \item \textbf{Normalizar dados:} Sub-processo automático executado durante o carregamento de ficheiros. Converte os dados para um formato estruturado através de uma \textit{pipeline} de normalização. Comunica com a base de dados e a aplicação de ficheiros.

    \item \textbf{Consultar gráficos:} Após o carregamento dos ficheiros, o utilizador pode consultar gráficos com base nos ficheiros carregados. A aplicação permite aplicar filtros, navegar entre \textit{quarters} e ver gráficos.

    \item \textbf{Criar conta:} Permite o registo de novos utilizadores. A aplicação valida os dados inseridos, verifica se o utilizador já existe e cria a conta, autenticando automaticamente o utilizador após sucesso.

    \item \textbf{Iniciar e Terminar sessão:} O utilizador introduz credenciais válidas (\textit{username} e \textit{password}) e, caso sejam corretas, é autenticado e redirecionado para a interface principal. Este processo inclui a gestão da sessão.

\end{itemize}

%%________________________________________________________________________
\section{Fundamentos}
\label{sec:fundamentos}

Neste capitulo iremos descrever as bases nas quais fundamentamos o projeto, e algumas noções necessárias para conseguir contextualizar as decisões tomadas.

\subsection{\textit{Marketplace Simulations}}
\label{sec:marketplace}

A \textit{Marketplace Simulations} \cite{MarketplaceSim_2025} é uma empresa que desenvolve plataformas de simulação para fins educativos, ou seja, ferramentas que colocam os estudantes numa espécie de jogo onde cada equipa gere a sua própria empresa e compete com os colegas em cenários simulados. Acabam então por funcionar como laboratórios virtuais que podem ser usados no ensino superior, onde os alunos aplicam os conceitos aprendidos numa experiência em contexto educativo.

No caso concreto do projeto, a aplicação em questão chama-se \textit{International Corporate Management} (referida neste relatório como plataforma de simulação), e é utilizada tipicamente no último semestre, na cadeira Projeto de Simulação em Negócios Internacionais da Licenciatura de Comércio e Negócios Internacionais \cite{FUC_ISCAL_2025}, que tem a interface apresentada na figura \ref{fig:marketplace1}.

\begin{figure}[h]
    \centering
    \includegraphics[max width=\textwidth]{./img/marketplace1}
 \caption{Captura de ecrã da aplicação \textit{International Corporate Management}}
 \label{fig:marketplace1}
 \end{figure}

Nesta plataforma, cada grupo de alunos representa uma empresa que tem de atuar num mercado internacional, tomando decisões sobre o posicionamento de produto, investimento, preços, distribuição, entre outras opções. Essas decisões são processadas pela plataforma, que simula o comportamento do mercado com base num algoritmo interno. 

A evolução temporal da simulação é dada por \textit{quarters}, que representam uma semana simulada. No final de cada \textit{quarter}, os alunos recebem os dados com os resultados das decisões anteriores, o que obriga a uma análise comparativa entre períodos. É este ciclo (decidir, analisar, ajustar, repetir) que dá sentido à simulação e aproxima o exercício a uma situação real.

A simulação está dividida em secções, cada uma representando uma área distinta do negócio gerido pelos alunos. Estas secções agrupam métricas, decisões ou outras informações relacionadas com aspetos específicos da empresa simulada, como por exemplo a procura por segmento, o desempenho financeiro, a perceção da marca ou a eficácia da equipa de vendas.

Cada secção representa também uma área onde os alunos têm de tomar decisões. As decisões variam consoante o tipo de secção, como por exemplo, na secção \textit{Customer Needs}, os alunos devem decidir que segmentos pretendem servir e com que marcas. Estas decisões são submetidas no final de cada \textit{quarter}, influenciando os resultados do seguinte periodo, que por sua vez geram novos dados para análise.

A plataforma \textit{International Corporate Management} permite exportar os dados de cada secção em ficheiros \textit{Excel}, sendo que cada ficheiro está associado a uma destas secções como por exemplo a secção\textit{Customer Needs}, entre outras.

Para a análise, os alunos têm à disposição os dados, na plataforma, nas várias secções disponiveis que na sua maioria são apresentados em tabelas, o que faz com que os alunos saltem entre secções, ou tenham de fazer gráficos à mão ou em último caso, extrair a informação da plataforma, não sendo então prático analisar a informação só pela plataforma de simulação.

Esta plataforma apresenta outros desafios, não é de código aberto pelo que não dá para perceber os seus algoritmos internos e só no contexto da unidade curricular é que se percebe o seu propósito, tornando dificil explicar o seu funcionamento fora do contexto da unidade curricular.

\subsection{\textit{Pipelines Extract, Transform Load}}
\label{ch:etl}

Uma \textit{pipeline} \gls{etl} é um processo utilizado em sistemas de tratamento e integração de dados, com o objetivo de mover dados de uma ou mais fontes para um destino, geralmente uma base de dados ou plataformas específicos para a análise de dados \cite{vassiliadis2009survey}. Uma \textit{pipeline} \gls{etl} é composta por três fases principais:

\begin{itemize}
  \item \textit{Extract} (Extrair): Consiste em recolher os dados das fontes de informação, que podem incluir bases de dados relacionais, APIs externas, sensores, entre outros. Esta fase preocupa-se com a capacidade de ler dados e de confirmar que são possíveis de extrair.  Exemplos de ferramentas que podem ser utilizadas nesta fase incluem o \textit{Apache NiFi} \cite{apache_nifi}, \textit{Fivetran} \cite{fivetran}, \textit{Airbyte} \cite{airbyte} e o \textit{Google Cloud Dataflow} \cite{dataflow}.
  
  \item \textit{Transform} (Transformar): Nesta fase os dados extraídos são processados e convertidos num formato mais adequado ao destino. Isso pode incluir tarefas como limpeza de dados (remoção de valores nulos ou duplicados), normalização, mudanças de tipo, ou cálculos. É nesta fase em que se garante a consistência da informação. Ferramentas populares para transformação incluem o \textit{dbt (data build tool)} \cite{dbt}, o \textit{Apache Beam} \cite{apache_beam} (usado com \textit{Dataflow}) e o \textit{Apache Spark} \cite{apache_spark}.

  \item \textit{Load} (Carregar): Por fim, os dados transformados são inseridos na aplicação de destino, normalmente um \textit{data warehouse},  O carregamento é feito dependendo dos requisitos da aplicação de destino. Exemplos de produtos que podem ser utilizados como destino são bases de dados como \textit{BigQuery} \cite{bigquery}, \textit{Snowflake} \cite{snowflake}, \textit{Clickhouse} \cite{clickhouse}, entre outros.

\end{itemize}

As \textit{pipelines} \gls{etl} são muito utilizadas em contextos onde há necessidade de consolidar dados de várias origens, permitindo análises mais completas. São fundamentais em áreas como \textit{business intelligence}, ciência de dados e integração de sistemas.


\subsection{Padrão Cliente-Servidor}

O projeto desenvolvido segue uma arquitetura de aplicações cliente-servidor, dividida em duas grandes camadas: \textit{backend} e \textit{frontend}.

\textbf{\textit{Backend}}

O \textit{backend} corresponde a ```parte invisível” da aplicação, ou seja, tudo o que acontece do lado do servidor. É a camada responsável por tratar os dados, executar a lógica de negócio e responder aos pedidos efetuados pelos utilizadores. No contexto específico deste projeto, o \textit{backend} é responsável por:

\begin{itemize}
    \item carregamento de ficheiros;
    \item processamento e normalização dos dados;
    \item autenticação e gestão de sessões;
    \item disponibilização de uma \gls{api} para o \textit{frontend} que permita acesso aos dados.
\end{itemize}

\textbf{Frontend}

O \textit{frontend} corresponde à ``parte visível'' da aplicação, ou seja, aquilo que o utilizador vê e com que interage no \textit{browser}. É a camada responsável por apresentar a informação de forma clara e permitir a interação com as funcionalidades disponibilizadas pela aplicação. No contexto deste projeto, o \textit{frontend} é responsável por:
\begin{itemize}
    \item os formulários, utilizados por exemplo para autenticação, criação de trimestres e envio de ficheiros;
    \item os gráficos que são apresentados ao utilizador, com os dados processados pelo \textit{backend};
    \item e outros elementos visuais criados a partir dos dados processados, como mensagens de erro, modais, entre outros.
\end{itemize}

Esta separação facilita a manutenção da aplicação e permite que ambas as partes sejam desenvolvidas de forma independente, podendo até ser substituídas sem necessidade de reescrever a aplicação completo. As tecnologias utilizadas para o desenvolvimento serão descritas no capítulo \ref{sec:tec}.

\subsection{Tipos de gráficos}

Durante a análise efetuamos foram estudados diferentes tipos de visualizações que podiam ser aplicadas, de acordo com os dados que tínhamos disponíveis. Cada tipo de visualização foi escolhido com base na sua capacidade de representar visualmente os dados e a sua facilidade de interpretação.  Consideramos então vários tipos de visualizações gráficas. 

Importa salientar que as imagens que acompanham esta secção foram retiradas diretamente da aplicação desenvolvida, já tendo em conta as limitações e as transformações realizadas nos dados. Estas transformações são discutidas mais a fundo no capítulo~\ref{sec:analiseInicial}.

\textbf{Gráfico de Barras (\textit{Bar Chart}):}

Este tipo de visualização foi utilizado para comparar valores entre diferentes categorias, como marcas, cidades ou segmentos de mercado. A disposição visual das barras permite uma leitura rápida das diferenças de desempenho entre categorias, sendo útil para dados que não são temporais. Foi também utilizado a variante barras agrupadas, dependendo se havia necessidade de comparar valores entre categorias.


\begin{figure}[H]
\centering
\includegraphics[max width=10cm]{./img/barras1}
\caption{Exemplo de gráfico de barras}
\end{figure}

\begin{figure}[H]
\centering
\includegraphics[max width=10cm]{./img/agrupada}
\caption{Exemplo de gráfico de barras agrupadas}
\end{figure}

\textbf{Gráfico de Sectores (\textit{Pie Chart}):}  

Utilizado para representar distribuições percentuais, como partições de dados por segmento. Este tipo é de gráfico é intuitivo para visualizar como uma totalidade se divide entre diferentes partes, sendo apropriado para dados onde se pretendia mostrar proporções relativas (como por exemplo dados categorizados por segmentos)

\begin{figure}[H]
    \centering
    \includegraphics[max width=10cm]{./img/pie}
    \caption{Exemplo de gráfico de sectores}
\end{figure}

\textbf{Gráfico Financeiro (\textit{Waterfall Chart}):}  

Utilizado especificamente para representar balanços financeiros, como valores acumulados de receitas e despesas. Permite também visualizar como diferentes contribuições individuais afetam um valor final, facilitando a análise de ganhos e perdas.

\begin{figure}[H]
    \centering
    \includegraphics[max width=10cm]{./img/waterfall}
    \caption{Exemplo de gráfico financeiro}
\end{figure}

\textbf{Diagrama de Quartis (\textit{Box Plot}):}

Os diagramas de quartis foram utilizados para representar dados que eram distribuições com mínimos e máximos, permitindo visualizar de forma prática a mediana, os quartis e valores extremos.  A interpretação deste tipo de visualização exige um conhecimento prévio de conceitos estatísticos como mediana e quartis, o que pode dificultar a leitura, mas que simplifica a representação dos dados.

\begin{figure}[H]
    \centering
    \includegraphics[max width=10cm]{./img/box}
    \caption{Exemplo de diagrama de quartis}
\end{figure}

\textbf{Diagrama de Sankey (\textit{Sankey Diagram}):}

Os diagramas de \textit{Sankey} são utilizados para representar fluxos entre diferentes categorias, sublinhando a quantidade transferida de uma categoria para outra. Cada fluxo é representado por uma linha cuja largura é proporcional à quantidade movida, permitindo uma visualização intuitiva da importância dos fluxos. Para evitar representações complexas, decidimos apenas recorrer a este tipo apenas quando os dados não poderiam ser representados de outra forma, ou tratados de forma a simplificar a representação.

\begin{figure}[H]
    \centering
    \includegraphics[max width=10cm]{./img/skankey}
    \caption{Exemplo de gráfico de \textit{Sankey}}
\end{figure}

Outros tipos de gráficos relevantes para o projeto, mas que não foram utilizados são:

\textbf{Gráfico de Linhas (\textit{Line Chart}):}  

Escolhido para representar séries temporais, como a evolução de uma variável ao longo do tempo. As linhas permitem identificar tendências e variações. No projeto, não tivemos necessidade de usar este tipo uma vez que nenhum dos dados recebidos era relativos a séries temporais, pelo que a sua utilização não era intuitiva para os utilizadores nem resultava numa visualização que fizesse sentido para os dados recebidos.

\textbf{Gráfico de Radar (\textit{Radar Chart}):}  

Utilizado para comparar múltiplas variáveis em relação a um valor comum, como no caso da avaliação de diferentes necessidades dos clientes em simultâneo. Este tipo de gráfico permite identificar rapidamente pontos fortes e fracos em várias dimensões de análise. Apesar destas vantagens, nos dados que recebemos não conseguimos identificar utilizações onde este tipo de gráfico beneficiasse os utilizadores finais.

\textbf{Gráfico de Dispersão (\textit{Scatter Plot}):}

Os gráficos de dispersão foram considerados para representar relações entre duas variáveis. Cada ponto no gráfico representa uma observação individual, permitindo identificar padrões de correlação (positiva, negativa ou inexistente) entre variáveis. No nosso caso, consideramos utilizar um tipo específico de gráfico de dispersão (gráfico de dispersão num mapa) mas a leitura tornou-se confusa, uma vez que era pouco legível para alunos, e não considerava localizações fictícias, fazendo com que a leitura fosse difícil.

A escolha dos tipos de gráficos teve como objetivo manter a clareza e facilidade da informação e permitir diferentes leituras sobre os dados extraídos. Cada gráfico foi pensado para às características dDOS dados disponiveis, considerando o formato dos dados (quantitativa ou categórica) e o objetivo da análise (comparação, distribuição, evolução ou composição).

%%________________________________________________________________________
\section{Abordagem}
\label{sec:abordagem}
%%________________________________________________________________________

O projeto proposto pretende então conseguir transformar dados desta plataforma em algo mais fácil e rápido de analisar. Tendo o contexto da plataforma, iremos então descrever duas abordagens que considerámos.

\subsection{\textit{Web \textit{scraping}}}

A primeira abordagem que consideramos foi a hipótese de automatizar a extração dos dados diretamente da plataforma do \textit{Marketplace Simulations}, através de técnicas de \textit{web} \textit{scraping}. A ideia parecia interessante numa fase inicial, já que permitiria reduzir a dependência do utilizador no processo de exportação  dos dados. No entanto, rapidamente percebemos que esta abordagem trazia vários desafios que, na prática, a tornavam pouco viável, ou mesmo arriscada.

Primeiro, cada conta na plataforma está associada a um grupo de alunos, ou seja, é uma conta ativa e única, usada diretamente durante a simulação. Isto significa que qualquer processo automático que iniciasse sessão, mesmo que fosse só para leitura, poderia de forma acidental interagir com a interface e acabar por alterar alguma opção, o que poderia comprometer a avaliação dos alunos porque podia afetar a sua avaliação final. Além disso, como o acesso à plataforma é feito por licenças pagas, não existe qualquer possibilidade de criar uma conta de serviço ou utilizador apenas para leitura. Ou seja, qualquer tentativa de \textit{web} \textit{scraping} teria de reutilizar credenciais reais, o que levanta não só questões de segurança, mas também (possivelmente) legais, uma vez que a plataforma pode não permitir o \textit{web} \textit{scraping}.

Outro fator que influenciou a nossa decisão foi o próprio risco técnico do \textit{web} \textit{scraping} uma vez que plataformas deste tipo estão muitas vezes protegidas com mecanismos contra automação (como por exemplo ecrãs CAPTCHA), e não conhecendo em detalhe a aplicação, poderíamos facilmente encontrar esse tipo de proteções, que são difíceis de automatizar.

Pelos motivos acima, optámos por não seguir esta abordagem. Em vez disso, definimos como parte da utilização da nossa aplicação que os próprios alunos devem exportar dados a partir da plataforma de simulação e carregar para a nossa aplicação. Esta solução, embora mais manual, garante segurança, respeita a integridade das contas dos utilizadores, e evita problemas legais ou técnicos com a aplicação de simulação.

\subsection{Exportação e carregamento manual de ficheiros}

A abordagem que acabamos por usar foi os alunos exportam manualmente os dados diretamente da plataforma de simulação e carregarem esses ficheiros na nossa plataforma. A partir daí, a aplicação processa e normaliza os dados recebidos, e mostra gráficos com base nesses dados. Esta abordagem, apesar de requerer uma ação manual do lado do utilizador, é mais segura que a alternativa de \textit{web} \textit{web} \textit{scraping}, pelos os motivos apresentados acima. Deste modo, garantimos um equilíbrio entre usabilidade, segurança e fiabilidade da aplicação. Acabamos então por criar um \gls{sad} (\cf, capítulo \ref{sec:sad}) que permita os estudantes tomarem melhores decisões.

Com esse objetivo, procurámos que a nossa aplicação refletisse a estrutura da própria simulação. Como tal, os dados são organizados por \textit{quarters}, como acontece na plataforma de simulação e permitir o carregamento dos dados exportados. Estes ficheiros são depois processados o que nos permite trabalhar com dados mais consistentes. 

Esta estrutura base implica a existência de três entidades principais na nossa aplicação (que iremos descrever nos capítulos seguintes) e cuja interação define  o funcionamento da aplicação.

\subsubsection{\textit{Quarters}}

Os \textit{quarters} funcionam como \textit{buckets} para organizar os ficheiros carregados pelos utilizadores. Cada utilizador pode criar múltiplos \textit{quarters}, identificados de forma única por um número. O propósito em incluir este conceito é para que a aplicação consiga associar os dados extraídos dos ficheiros carregados ao \textit{quarter} correspondente sem depender de manipulações do nome do ficheiro carregado, e desde modo assegurar que a plataforma consegue identificar corretamente os \textit{quarters}. 

A desvantagem é que requer \textit{input} do utilizador para que seja criado ou atribuído o \textit{quarter} correto. De modo a facilitar a criação de \textit{quarters}, tentamos que experiência de utilização fosse centrada no carregamento de ficheiros, limitando os \textit{quarters} a um campo obrigatório no formulário de carregamento auto-preenchido, ou seja, tornando implícita a criação de \textit{quarters} no momento de carregamento de ficheiros.

\subsubsection{Ficheiros}

Os ficheiros são inicialmente carregados no formato \gls{xlsx} (que é o formato que a plataforma de simulação exporta), contendo uma ou várias folhas de cálculo. Cada folha é tratada como uma fonte de dados individual e é de onde são extraídos os dados. O ficheiro \gls{xlsx} é guardado como referência, mas não é diretamente utilizado para visualização, ficando apenas como artefacto interno da aplicação. O modelo proposto apenas considera um gráfico por folha de cálculo.

O processamento que é aplicado é uma \textit{pipeline} de transformação de dados (conhecido na indústria como \gls{etl} (\cf, capítulo \ref{ch:etl})), que aplica regras para que os gráficos possam ser mostrados de forma consistente. Os dados resultantes desta transformação estão associados ao ficheiro \gls{xlsx} original, sendo até possível voltar a processar ficheiros de forma a recriar informação (esta funcionalidade não é disponibilizada aos utilizadores finais e não foi desenvolvida de forma explícita).

A aplicação garante que só existe uma versão ativa de cada ficheiro. Caso o utilizador carregue novamente um ficheiro com o mesmo nome, o anterior será marcado como não ativo, evitando duplicações e garantindo que os gráficos usam apenas dados mais recentes, e em caso de remoção, garante que conseguimos reverter para a versão anterior.

\subsubsection{\textit{Pipeline} de Transformação de Dados}

Para conseguirmos então garantir uma experiência de utilização consistente, desenvolvemos uma \textit{pipeline} transformação de dados, semelhante a uma \textit{pipeline} \gls{etl} (\cf, capítulo \ref{ch:etl}), ainda que neste projeto tenha sido desenvolvido com uma escala mais pequena, e com recurso a bibliotecas de processamento de dados. 

O objetivo é garantir que os ficheiros carregados, que muitas vezes contêm nomes de colunas inconsistentes, quebras de linha, espaços em excesso ou colunas irrelevantes, sejam adaptados para serem visualizados, e que o processamento seja determinístico.

Como podemos receber muitos ficheiros, a variabilidade entre os dados recebidos é muito alta, pelo que alguns dados passam por mais do que uma fase de transformação. Esta decisão foi tomada com base numa análise manual, em que identificámos possíveis fontes de dados que precisam de mais do que uma fase de transformação. As várias fases de transformação alteram os dados de modo a facilitar a representação visual dos mesmos e é um passo essencial no projeto, porque garante que a aplicação trabalha com formatos e regras conhecidas, e remove a variabilidade dos ficheiros importados. Para proteger a aplicação, decidimos apenas mostrar os dados que foram extraídos de ficheiros conhecidos, de modo a garantir a consistência dos dados mostrados.

As fases de transformação irão ser descritas em mais detalhe nos capítulos seguintes, uma vez que a implementação destas pipeline estão relacionadas à tecnologia escolhida, mas o desenvolvimento desta \textit{pipeline} é um fator diferenciador deste projeto, uma vez que tem de lidar com dados que não estão estruturados de forma a facilitar representações visuais. 

\subsubsection{Utilizadores}

A aplicação foi projetada para funcionar com utilizadores. Cada utilizador tem a sua conta, e pode criar \textit{quarters}, carregar ou alterar ficheiros, e ver aos gráficos criados a partir desses ficheiros.

Apesar da aplicação não suportar explicitamente grupos, assume-se que alunos do mesmo grupo podem carregar ficheiros semelhantes, mas a aplicação trata-os como ficheiros diferentes. Assim, evita-se a complexidade adicional de gerir permissões ou partilha de dados entre contas.  Também se assume que as contas podem ser criadas ao nível do grupo, pelo que para a aplicação, é indiferente se a conta é individual ou partilhada entre membros desse grupo.

Cada utilizador tem acesso apenas aos seus próprios dados, garantindo o isolamento da informação. Esta separação é feita a nível da base de dados, através da associação de cada entidade (ficheiro ou \textit{quarter}) ao utilizador que criou.

No futuro, pode ser considerada a funcionalidade de desativação automática de contas (por exemplo, após o final do semestre) ou a gestão das contas por um docente da unidade curricular.

No próximo capitulo iremos então descrever as escolhas tomadas de forma a concretizar este modelo.

\section{Arquitetura e Tecnologias Adotadas}
\label{sec:tec}

\begin{figure}[h]
\centering
\includegraphics[max width=\textwidth]{./img/arch}
\caption{Arquitetura da aplicação}
\label{fig:arquitectura}
\end{figure}

A arquitetura final da aplicação pode ser observada na Figura~\ref{fig:arquitectura}. Esta segue o modelo cliente-servidor, onde os utilizadores interagem com a aplicação através de um \textit{browser}. A escolha das tecnologias baseou-se na familiaridade prévia com cada uma, na maturidade da mesma, as respetivas comunidades, bem como o contexto e requisitos do projeto. Nas seguintes secções especificamos que tecnologias escolhemos e que funções desempenham na aplicação.

\subsection{Django}

Para o desenvolvimento do \textit{backend} da aplicação, optámos por usar a \textit{framework} \textit{Django}. A escolha deveu-se ao facto de o \textit{Django} ser uma \textit{framework} muito usada, com uma arquitetura conhecida e uma comunidade muito grande.

Uma das principais vantagens do \textit{Django}\cite{djangodocs} é o facto de seguir uma variante do padrão \gls{mvc}, conhecida como \gls{mtv},  o que facilita bastante a organização da aplicação e a separação de responsabilidades.

\begin{itemize}
    \item o \textit{Model} representa a camada de dados e corresponde ao modelo do domínio;

    \item o \textit{Template} representa a camada de apresentação (interface com o utilizador);

    \item a \textit{View} é a camada intermediária que processa a lógica da aplicação e interage com as outras camadas.
\end{itemize}

Para além disso, o \textit{Django} utiliza um \gls{orm}, que permite interagir com bases de dados sem necessidade de escrever \gls{sql} manualmente. Este \gls{orm} mapeia os modelos para entidades na base de dados e suporta operações como filtros, agregações e relações entre tabelas \cite{djangodocs}, e ao mesmo tempo previne falhas de segurança, como \textit{SQL injection}, através do uso de consultas parametrizadas \cite{kumar2016security}.

Outra vantagem é a as funcionalidades gestão de utilizadores e sessões já estar incluído no \textit{framework}. Isto permite ao programador focar-se no desenvolvimento das funcionalidades específicas da aplicação sem ter de construir esse suporte de raiz.

Outro fator que consideramos positivo foi a integração com outras bibliotecas \textit{Python}, como o \textit{Pandas}, que usamos para o processamento de dados. Esta compatibilidade torna o desenvolvimento mais rápido e flexível, reduzindo o esforço necessário (e código) para ligar diferentes tecnologias.

Em alternativa, podíamos ter escolhido bibliotecas como \textit{Flask} \cite{grinberg2018flask} ou \textit{FastAPI} \cite{tiangolo2018fastapi}. Apesar de \textit{FastAPI} ser mais recente e mais rápido para \gls{api} \gls{rest}, o nossa aplicação, focado na transformação de dados e visualização, não necessitava de uma abordagem completamente assíncrona e não funciona totalmente com \gls{api} \gls{rest}. Já o \textit{Flask}, apesar de ser mais leve, não vem com funcionalidades como gestão de utilizadores e criação de contas, pelo que teríamos de desenvolver esses mecanismos de raiz.

Relativamente a outras opções como \textit{WordPress}, excluímos essa hipótese. O \textit{WordPress}, que é uma plataforma que suporta sites de conteúdo como blogues ou sites de notícias, não é adequado para projetos que exigem muito controlo sobre a estrutura dos dados e não integra bem com as outras tecnologias escolhidas.

\subsection{Pandas}

Para o processamento de dados, a biblioteca que decidimos usar foi o \textit{Pandas}. Esta escolha foi motivada pelo facto de ser muito utilizada em projetos de engenharia de dados, e por ser escrita em \texit{Python}.

O \textit{Pandas} permite ler e transformar ficheiros \gls{xlsx}. A sintaxe e as funcionalidades disponibilizadas para operações como normalização, filtragem, agrupamento e ordenação tornaram esta biblioteca adequada às necessidades do projeto. Uma vantagem adicional é a capacidade de converter os dados para formatos como \gls{json}, o que facilitou a sua integração com o restante da aplicação.

Apesar destas vantagens, o \textit{Pandas} não é a melhor opção se considerarmos \textit{datasets} muito grandes, uma vez que funciona inteiramente em memória. Em contextos com maiores volumes de dados, poderiam ser consideradas bibliotecas como \textit{Dask} ou \textit{Polars} por terem capacidades de paralelismo para lidar com um maior conjunto de dados (acabam por ser mais adequadas para \textit{BigData}). No entanto, para os objetivos e escala deste projeto, o \textit{Pandas} é a escolha mais equilibrada considerando o \textit{scope} do projeto que desenvolvemos.

\subsection{Plotly}

Para mostrar os gráficos da aplicação, optámos por utilizar o \textit{Plotly}, que é uma biblioteca  que permite criar visualizações, e que está disponível em várias linguagens como \texit{Python} e \texit{Javascript}. Esta escolha foi motivada pelo facto de o \textit{Plotly} suportar suportar uma \texit{API} declarativa, o que torna mais prática a integração com o \textit{frontend} e \textit{backend}, porque assim a configuração dos gráficos pode ser retornada pelo servidor, permitindo uma configuração mais estável e garantindo que todos os utilizadores vêm o mesmo tipo de gráfico. Internamente, o \textit{Plotly} utiliza uma outra biblioteca, \textit{D3.js}, que é uma biblioteca com capacidade de criar gráficos personalizados.

Além disso, a biblioteca oferece suporte a uma grande variedade de gráficos, desde gráficos de barras até formatos mais especializados como mapas de calor, o que foi importante para explorar os vários tipos de visualização poderíamos usar. Em comparação, bibliotecas como \textit{Chart.js}, embora sejam mais leves, não oferecem a mesma flexibilidade e variedade de visualizações. Por este motivo, \textit{Plotly} era a solução mais adequada para a flexibilidade e diversidade de visualizações que queríamos. O \textit{Plotly} permite criar versões da biblioteca com gráficos específicos, reduzindo bastante o tamanho do ficheiro \textit{Javascript}. Uma lista completa das diferenças pode ser consultada no capítulo \ref{ch:charts}.

A versão \textit{Python} desta biblioteca foi também importante no desenvolvimento dos gráficos, uma vez que nos permitiu explorar os dados fora da aplicação, de modo a escolher que tipo de gráfico seria o mais adequado aos dados que estávamos a analisar e que transformações eram necessárias. Este processo de exploração permitiu-nos perceber o que era possível fazer, e com a ajuda dos orientadores, definir estratégias alternativas para os conjuntos de informação que eram mais difíceis de mostrar.

\subsection{\textit{WebComponents}}

Para conseguirmos isolar a implementação dos gráficos decidimos utilizar \textit{WebComponents} \cite{webcomponents}. Esta decisão teve como base a necessidade de manter uma boa experiência e criar componentes que podiam ser reutilizados para os vários tipos de visualização. 

Os \textit{WebComponents} \cite{webcomponents} são uma especificação nativa dos \textit{browsers}, que permite definir componentes reutilizáveis com encapsulamento de estilo e comportamento. Ao evitarmos dependências pesadas como o \textit{React}, conseguimos reduzir a complexidade técnica da aplicação, mantendo ao mesmo tempo a flexibilidade e capacidade de reutilização dos componentes desenvolvidos.

Apesar de \textit{React} ser uma biblioteca muito utilizada para aplicações \textit{web}, considerámos que para este projeto, a sua introdução seria aumentar as dependências e complexidade. A alternativa escolhida é uma solução mais leve e mais fácil de manter. 

Outra razão que nos levou a esta escolha foi o facto do \textit{Django} já trazer consigo as funcionalidades de criação e gestão de contas de utilizador1, pelo que se a aplicação \textit{web} fosse escrita totalmente em \textit{React}, teríamos de usar o \textit{Django} apenas como uma \gls{api} \gls{rest}, e utilizar métodos alternativos de criação e gestão de contas de utilizador1 como \textit{OAuth2}.

\subsection{\textit{PostgreSQL}}

Para a persistência de dados, utilizámos a aplicação de gestão de bases de dados \textit{PostgreSQL} que é uma das soluções \textit{open-source} mais completas atualmente. Inicialmente foi considerada a utilização de \textit{SQLite} por simplicidade durante o desenvolvimento, mas como iriamos usar \gls{json} para armazenar dados,  decidimos que o \textit{PostgreSQL} era a melhor opção. O \textit{PostgreSQL} têm um bom desempenho, suporte a consultas mais complexas e é compatível com o \gls{orm} do \textit{Django}.

\subsection{Docker}

O \textit{Docker} foi utilizado neste projeto, tanto para desenvolvimento como para ambientes de produção. A principal vantagem é a criação de ambientes de execução consistentes e isolados, garantindo que a aplicação corre da mesma forma em diferentes máquinas, sem conflitos de dependências ou configurações, e consistente em vários sistemas operativos. 

Existem vários \textit{runtimes} compatíveis com \textit{Docker}, e devido ao modelo de negócio da empresa \textit{Docker} (que adotou um modelo pago), optámos por usar o \textit{Rancher} (uma aplicação com uma \gls{api} compatível com \textit{Docker}) para o desenvolvimento, e o \textit{Docker Engine} para ambientes de produção.

Durante o desenvolvimento, o \textit{Docker} facilitou a orquestração dos vários serviços usados (a aplicação \textit{Django}, a base de dados, e o servidor \gls{http} \textit{Nginx} configurado como \textit{reverse proxy}), através de ficheiros \texttt{Dockerfile} e \texttt{docker-compose.yml}. Isto permitiu-nos montar o ambiente da aplicação, sem necessidade de instalações manuais.

Para produção, o \textit{Docker} permite recriar um ambiente de produção sempre com a mesma configuração e gerir os vários serviços que a aplicação utiliza, o que facilita o processo de \textit{deploy}, e escalabilidade futura. Esta abordagem garante que o código testado é exatamente o que será executado em produção, reduzindo erros e aumentando a estabilidade da aplicação.

O \textit{Docker} foi também utilizado durante o desenvolvimento deste relatório, porque permitiu-nos ter um ambiente \textit{LaTex} configurado com todas as extensões apenas com um único comando no terminal sem instalar aplicações.

\section{Outras ferramentas utilizadas}
\label{sec:tools}

Além das tecnologias centrais utilizadas no projeto, foi também fundamental a utilização de um conjunto de ferramentas de suporte que facilitaram o trabalho, a organização e o desenvolvimento do projeto. As principais ferramentas utilizadas foram:

\begin{itemize}
    \item \textbf{Git \cite{git} e Github \cite{github}}: Para assegurar o controlo de versões, foi utilizado a aplicação \texit{Git}, em conjunto com a plataforma \textit{Github}. Esta combinação permitiu não só manter um histórico das alterações feitas e reverter alterações que podiam ter introduzido defeitos na aplicação

    \item \textbf{Notion \cite{notion}}: A rganização do trabalho foi feita na plataforma \textit{Notion}. Esta aplicação é prática de fácil de usar e permitiu-nos para planear as tarefas e acompanhar o progresso das diferentes fases do projeto e também tirar notas durante o desenvolvimento, que foram importantes para o desenvolvimento do relatório.

    \item \textbf{\gls{vscode}\cite{vscode}}: Como editor, foi utilizado o editor \gls{vscode}. Através da instalação de extensões, foi possível adaptar o \gls{vscode} às necessidades do projeto, melhorando a experiência de programação.
\end{itemize}

Estas ferramentas foram importantes no desenvolvimento do projeto, não só do ponto de vista técnico, mas também no que respeita à organização e planeamento do mesmo.

%%________________________________________________________________________
%% LEIM | PROJETO
%% 2022 / 2013 / 2012
%% Modelo para relatório
%% v04: alteração ADEETC para DEETC; outros ajustes
%% v03: correção de gralhas
%% v02: inclui anexo sobre utilização sistema controlo de versões
%% v01: original
%% PTS / MAR.2022 / MAI.2013 / 23.MAI.2012 (construído)
%%________________________________________________________________________


%%________________________________________________________________________
\chapter{Implementação do Modelo}
\label{ch:implementacaoDoModelo}
%%________________________________________________________________________

Neste capítulo iremos descrever de forma detalhada como foi feita a transição do modelo proposto para a implementação  da aplicação. A abordagem seguida foi iterativa, com várias fases de experimentação e validação, sempre em articulação com os objetivos definidos para o projeto e os dados reais obtidos anteriormente.

Começamos por abordar o processo de análise e classificação dos dados, uma vez que a estrutura e qualidade da informação recebida tiveram um impacto direto na forma como os dados seriam processados, armazenados e, mais tarde, visualizados. A seguir, exploramos a construção do pipeline de transformação, a lógica de organização dos ficheiros e o modo como garantimos que cada carregamento fosse tratado de forma automática. Por fim, explicamos as decisões que nos levaram à escolha de representações gráficas específicas, assim como os mecanismos que permitem ao utilizador interagir com os dados.

\section{Análise e Transformação dos dados}

O processo de análise começou após a reunião inicial com os orientadores do projeto, onde recebemos os ficheiros com que iamos trabalhar. Tratar este ficheiros implicou várias fases de análise, onde começamos a pensar em estratégias para conseguir extrair dados e possiveis gráficos, e tornar este processo modular e automatizado.

\subsection{Análise inicial dos dados}
\label{sec:analiseInicial}

Após termos recebido os dados, fizemos uma primeira análise com o objetivo de perceber a informação recebida e os próximos passos a tomar. No total, recebemos 30 ficheiros \gls{xlsx} que foram exportados da plataforma, correspondentes às diferentes secções da plataforma de simulação, com vários indicadores como segmentos de mercado, características de produto, vendas, entre outros. Cada ficheiro pode conter várias folhas de cálculo, que no total são 103 folhas de cálculo, organizadas por marcas, cidades entre outros marcadores.

Durante esta análise, percebemos que:

\begin{itemize}
    \item Os nomes dos ficheiros e das folhas variam, mas seguem uma estrutura relativamente consistente;
    \item Alguns ficheiros apresentam estruturas de dados semelhantes, mas com diferenças no nível de detalhe ou na organização das colunas e linhas;
    \item Existiam ficheiros com dados que iriam precisar de mais transformação uma vez que representavam vários tipos de unidades (como por exemplo percentagens, valores monetários, valores relativos, etc) na mesma folha de cálculo;
    \item Dados com muito detalhe, e com várias colunas sem representação (células marcadas com "X", linhas com valores nulos e fundo amarelo).
\end{itemize}

Com base nesta primeira análise, concluímos que seria necessário:
\begin{itemize}
    \item Fazer uma normalização da informação recebida para garantir uma utilização consistente da aplicação;
    \item Separar os vários ficheiros possiveis em grupos, de forma a identificar dados em comum que pudéssemos aplicar \textit{templates} de gráfico;
    \item Guardar os dados extraídos num formato mais prático, e que permita uma filtragem dinâmica da informação carregada (de forma a facilitar a visualização dos dados).
    \item Em algumas séries de dados, identificamos linhas e colunas que se podiam remover devido a serem redundantes, ou por representarem informação que já é representada na mesma folha (como por exemplo colunas de valores totais, linhas que repetiam as marcas, ou outras discutidas com os orientadores);
\end{itemize}

Esta primeira análise serviu essencialmente para estruturar o modo como iríamos tratar os diferentes ficheiros que os utilizadores podem submeter, ou seja, deu-nos uma base para sistematizar como iriamos transformar a informação de forma lógica.

Com isso em mente, optámos por agrupar a informação em várias categorias, com base na natureza e estrutura dos dados de cada folha de cálculo. A cada uma dessas categorias passámos então a associar um tipo de gráfico específico, o que nos permitiu criar uma espécie de conjunto de templates reutilizáveis. Esta abordagem facilita bastante o processo, porque conseguimos aplicar esses templates de forma programática, sem precisar de decisões manuais ficheiro a ficheiro.

\subsection{Classificação dos dados}

A análise inicial dos ficheiros fornecidos permitiu-nos perceber que, apesar da informação ser diferente,  existiam padrões na forma como estava estruturados. Com base nisso, tomámos a decisão de agrupar os ficheiros em \textit{buckets} ou categorias. Cada um desses grupos ficou associado a um \textit{template} gráfico específico, o que nos permite não só uniformizar a apresentação dos dados, como também automatizar o processo de transformação e visualização a partir dos ficheiros.

Os grupos definidos foram os seguintes: \textbf{simples}, \textbf{duplo}, \textbf{balanços}, \textbf{setores} e \textbf{análise específica}. Vamos então descrever o significado de cada grupo e identificar o que é comum entre os gráficos pertencentes a cada um deles.

O grupo \textbf{simples} inclui ficheiros em que a estrutura é mais direta, geralmente com apenas duas colunas: uma coluna que representa uma categoria (como por exemplo empresas ou segmentos de mercado) e uma coluna numérica. Nestes casos, optámos por gráficos de barras, uma vez que o objetivo principal é comparar rapidamente valores individuais entre categorias. Nos ficheiros que avaliamos, não encontramos séries temporais, pelo que não justificou gráficos de linhas. Também nesta categoria, alguns dados eram séries mais específicas (como dados relativos a médias e medianas) mas mantinham a mesma estrutura de dados, pelo que foram classificados como \textbf{simples} mas utilizaram gráficos diferentes como gráficos de diagramas e quartis e gráficos de setores (também conhecidos como \textit{pie charts}).

O grupo \textbf{duplo} refere-se a ficheiros onde existem múltiplas séries de dados associadas à mesma categoria. Ou seja, para cada categoria, existem vários valores (ou várias séries) que precisam de ser representados lado a lado. Para estes casos, a escolha que fizemos foi apresentar gráficos de barras agrupadas e barras empilhadas, permitindo uma comparação direta entre diferentes séries de dados para a mesma categoria, e permite visualizar todos os dados relacionados com essa categoria.

O grupo \textbf{balanços} abrange ficheiros associados a séries de dados financeiros, onde faz sentido representar aumentos e diminuições de valores ao longo de um processo ou período. Para estes ficheiros, o \textit{template} selecionado foi o gráfico de cascata (ou um gráfico \textit{financial waterfall}), dado que este tipo de visualização representa as várias componentes do valor total numa forma fácil de perceber.

O grupo \textbf{setores} inclui ficheiros onde a informação está agrupada em em duas e três colunas (como por exemplo dados por empresa, marca e cidade). Nestes casos, faz sentido usar gráficos de setores e barras agrupadas,  dependendo da necessidade de comparar proporções entre segmentos.

Finalmente, o grupo de \textbf{análise específica} são ficheiros que não se encaixam diretamente nos formatos anteriores, como análises mais detalhadas por cidade, por segmento ou quando os dados incluem várias unidades na mesma folha (percentagens com euros). Para estes casos, o processo passou por simplificar a informação de modo a encaixar num dos grupos acima ou excecionalmente aplicar um novo tipo de gráfico, e a análise foi feita manualmente para cada caso. Para casos em que a não era possível aplicar um \textit{template} de gráfico, não foi possivel simplificar a informação e não encontrámos uma representação gráfica que fosse fácil de interpretar, optámos por não aplicar nenhum gráfico, e apenas apenas mostrar uma tabela interativa com os dados onde o utilizador podia filtrar os dados por diferentes colunas.

No final, dos 30 ficheiros recebidos, a classificação foi a seguinte:

\begin{figure}[h]
    \centering
    \includegraphics[max width=\textwidth]{./img/stats1}
 \caption{Classificação dos ficheiros - contagem final dos grupos}
 \end{figure}

Em termos de representações utilizadas, a distribuição é a seguinte:
\begin{figure}[h]
    \centering
    \includegraphics[max width=\textwidth]{./img/stats2}
 \caption{Classificação dos ficheiros - contagem final de representações utilizadas}
 \end{figure}

\subsection{Transformação dos dados}

Com a classificação de dados feita, foi necessário desenvolver um \textit{pipeline} de processamento que permitisse normalizar e transformar os dados dos ficheiros em um formato mais adequado para visualização e para guardar na base de dados. Este processo foi desenvolvido com recurso a biblioteca Pandas (Python), que oferece métodos para manipulação e transformação de dados. Esta \textit{pipeline} funciona de forma assincrona, e é chamada pelo \textit{backend} quando um ficheiro é carregado, de forma a não bloquear a interface do utilizador.

A \textit{pipeline} de processamento foi dividida em duas fases, cada uma com um objetivo específico na transformação dos dados.

\subsection{Extração e Normalização Inicial}

Nesta fase, os dados são extraídos dos ficheiros \gls{xlsx} e normalizados para um formato consistente:

\begin{itemize}
    \item Extração do título do gráfico a partir da primeira linha de cada folha
    \item Identificação de cada folha por num nome único (derivado do nome da folha de cálculo), para que possamos identificar os dados de forma única na aplicação (ou seja, um \textit{slug} que mapeia exclusivamente a uma série de dados). Este passo é importante para garantir que cada série de dados é tratada de forma única e consistente, e para conseguirmos manter uma configuração para cada série de dados.
    \item Normalização dos cabeçalhos das colunas (remoção de quebras de linha, espaços extras)
    \item Remoção de colunas baseadas na análise feita na secção \ref{sec:analiseInicial}
    \item Normalização dos dados nas células (remoção de quebras de linha, espaços duplos)
    \item Tratamento de valores nulos e vazios
    \item Normalização de valores numéricos com precisão configurável
    \item Manutenção da ordem original das colunas para preservar a estrutura dos dados. A biblioteca Pandas automaticamente ordena as colunas, pelo que é necessário manter a ordem original das colunas para preservar a estrutura dos dados.
    \item Transformação de dados marcados como "X" em formato binário (0/1) para indicar presença/ausência de valores
    \item Normalização de valores decimais num valor escolhido pelo utilizador (com o valor por omissão a 9)
\end{itemize}

\subsection{Processamento Específico por Tipo}

Após a normalização inicial, os dados passam por transformações específicas baseadas no seu tipo, e no tipo de gráfico que pretendemos representar:

\begin{itemize}
    \item Aplicação de transformações específicas baseadas no tipo de gráfico identificado (financeiro, percentagens, valores relativos entre outros)
    \item Processamento para balanços financeiros, incluindo cálculos de percentagens e valores relativos de outros valores (aplicar somas e subtrações como descrito na folha de cálculo)
    \item Transformação dos dados para conseguir representar categorias e a relações entre elas (como por exemplo a relação entre marcas e segmentos de mercado e cidades)
    \item Para folhas que continham multiplos \textit{quarters}, optámos por escolher o \textit{quarter} para o qual a folha foi carregada, sendo que essa configuração pode ser trocada em código.
\end{itemize}

De forma a tornar modular toda a lógica de normalização de dados, decidimos criar configurações na aplicação que mapeiam ficheiros a gráficos e as suas transformações e outras configurações associadas a colunas que são usadas para representar os dados, sendo que cada série de dados é identificada unicamente pelo seu \textit{slug}. Esta configuração não é guardada na base de dados, e só pode ser editada em código, porque os dados extraidos, para serem mostrados, precisam de ser processados de forma especifica de modo a que consigam ser representados, e a sua visualização foi pensada para ser um gráfico especifico, pelo que não era possivel suportar várias representações para a mesma série de dados. Após os dados serem processados, já não é possivel alterar a visualização dos dados, porque o processamento adapta a informação que recebe para a visualização pretendida.

\subsection{Conversão e Armazenamento em Base de Dados}
\label{sec:armazenamentoDados}

Após o processamento dos dados ser feito, os mesmos são guardados como \gls{json} num campo especifico \textit{JSONField} da base de dados que decidimos utilizar, PostgreSQL. Estes dados guardados tentam ser os mais próximos ao dados mostrados ao utilizador, mas sem perder informação quando contem várias séries.

Para garantir a consistência dos dados e evitar conflitos de ficheiros com o mesmo nome a serem carregados para o mesmo \textit{quarter}, foi implementado um mecanismo que permite marcar ficheiros como processados / não processados e como correntes / não correntes. Quando um utilizador carrega um ficheiro, este é marcado como processado e como corrente, e os dados são guardados na base de dados. Quando um utilizador carrega um novo ficheiro com o mesmo nome, o novo é marcado como corrente, e o antigo é marcado como não corrente. Em nenhum momento é apagado versões anteriores dos dados nem os ficheiros originais que foram carregados. A aplicação tem depois um mecanismo em que só considera ficheiros marcados como correntes para serem mostrados ao utilizador.

Isto permite recuperar versões anteriores dos dados quando o utilizador apaga um ficheiro ou quando existe algum erro no processamento, uma vez que os dados resultantes só são marcados como correntes quando são extraidos e processados com sucessos.

Este sistema de processamento permitiu transformar os dados ficheiros \gls{xlsx} em um formato estruturado e normalizado, facilitando a visualização e análise dos dados através da aplicação web. A modularidade da \textit{pipeline} também permitiu adicionar passos de normalização conforme necessário, mantendo a consistência dos dados processados.

No final, obtemos dados que são práticos de representar, próximos ao formato final que pretendemos mostrar ao utilizador, e que podem ser facilmente utilizados pela aplicação web para mostrar os gráficos.

\begin{figure}[H]
\centering
\includegraphics[max width=\textwidth]{./img/before}
\caption{Excerto de uma folha de cálculo recebida}
\end{figure}

\begin{figure}[H]
\centering
\includegraphics[max width=\textwidth]{./img/after}
\caption{Gráfico extraido da mesma folha acima com filtro aplicado para o Quarter 5}
\end{figure}

\section{Desenvolvimento da Aplicação Web}

Esta secção foca-se na estrutura e funcionamento da aplicação web, tanto o \textit{backend} em Django como a interface construída com \gls{html}, \textit{WebComponents} e Flowbite.

\subsection{Arquitetura da aplicação (Backend)}

A aplicação web foi desenvolvida utilizando o framework Django como já descrito na secção \ref{sec:fundamentos}. A arquitetura foi desenhada para garantir o isolamento dos dados por utilizador e uma gestão dos ficheiros e dados processados.

\subsubsection{Modelos de Dados}

A aplicação utiliza três modelos principais para gerir os dados, cada um com um papel específico na gestão e organização da informação. Estes modelos formam a base da aplicação, permitindo uma estrutura clara e organizada dos dados. Como o Django utiliza o \gls{orm} para gerir as relações entre os modelos, cada classe representa também um modelo na base de dados, que é apresentado na figura \ref{fig:er-diagram}. A classe \textit{User} é uma classe que já vem incluida com o Django, e é utilizada para gerir os utilizadores e sessões da aplicação.

\begin{figure}[H]
    \centering
    \includegraphics[max width=\textwidth]{./img/er-diagram.png}
 \caption{Diagrama de entidade-relação da aplicação}
 \label{fig:er-diagram}
 \end{figure}


O nosso modelo de dados é composto por três modelos principais: \textit{Quarter}, \textit{ExcelFile} e \textit{ChartData}, cada um com um papel específico no contexto da aplicação.

\begin{itemize}
    \item \textbf{Quarter:} Representa um trimestre específico para um utilizador. Cada \textit{quarter} tem:
    \begin{itemize}
        \item Um número único por utilizador
        \item Uma precisão configurável para valores numéricos (por omissão 9 casas decimais)
        \item Um \gls{uuid} único para identificação
        \item Relação com o utilizador que o criou
    \end{itemize}

    \item \textbf{ExcelFile:} Representa um ficheiro Excel carregado para um \textit{quarter} específico:
    \begin{itemize}
        \item Armazena o ficheiro físico em pastas organizadas por \gls{uuid}
        \item Mantém o estado de processamento (processado/não processado)
        \item Guarda metadados como a secção do ficheiro
        \item Relaciona-se com o \textit{quarter} e o utilizador
        \item Controla qual versão dos dados está ativa (corrente ou não)
    \end{itemize}

    \item \textbf{ChartData:} Armazena os dados processados de cada folha do Excel:
    \begin{itemize}
        \item Guarda os dados em formato \gls{json}
        \item Mantém a ordem original das colunas
        \item Guarda metadados como o nome da folha e o \textit{slug} para identificar os dados unicamente
        \item Relaciona-se com o ficheiro Excel de origem e o utilizador
    \end{itemize}
\end{itemize}

A utilização de identificadores \gls{uuid} foi importante uma vez que, para além de identificar unicamente cada instancia da entidade, faz com que não seja possivel aceder aos dados só a aumentar o identificador, como se fosse o caso se o identificador fosse um número inteiro sequencial. O Django já disponibiliza uma chave primária em cada modelo por omissão, que é um número sequencial inteiro.

\subsubsection{Gestão de Utilizadores}

A aplicação utiliza o mecanismo de utilizadores que já vem incluido com o Django. Este sistema oferece uma camada de segurança, garantindo que apenas utilizadores autenticados possam aceder aos dados. A autenticação na aplicação é feita através de nome de utilizador e palavra-passe e as sessões são geridas pelo Django.

Para garantir a segurança dos dados, todas as rotas que requerem autenticação estão decoradas com \texttt{@login\_required}. Além disso, a aplicação isola dados dados por utilizador, garantindo que cada utilizador só pode aceder aos seus próprios. Este isolamento é implementado através de filtros nas pesquisas, que são aplicados automaticamente em todas as operações de leitura e escrita que são feitas.

Os utilizadores podem criar conta na aplicação, sendo que podem criar conta para si ou para um grupo de utilizadores. A aplicação não faz distinção entre uma conta para um utilizador ou para uma conta para multiplos utilizadores.

\subsubsection{Gestão de Ficheiros e Processamento de dados}

Para conseguirmos manter vários ficheiros em disco, tivemos de pensar num mecanismo capaz de manter ficheiros organizados. Para isso, desenvolvemos um mecanismo que garante que os dados carregados pelos utilizadores ficam organizados em pastas isoladas. Em vez de depender apenas do nome do ficheiro (o que podia facilmente gerar conflitos ou sobreposições de nomes), a aplicação cria automaticamente uma pasta associada a um \gls{uuid}, o que garante que cada ficheiro fica isolado do resto. Isto evita situações em que dois ficheiros com nomes iguais se sobrepõem e, ao mesmo tempo, facilita bastante a estruturação interna por \gls{quarter}. Assim que o ficheiro é carregado, o processamento é iniciado, sem depender de passos manuais , ou seja, o utilizador só precisa de fazer o \textit{upload} e o resto é tratado pela aplicação.

Para além disso, a aplicação guarda registos de todos os ficheiros carregados. Quando se carrega um novo ficheiro para substituir outro com o mesmo nome ou que resulta nos mesmos dados, a aplicação não apaga o anterior, como já foi descrito no capítulo (\cf, capítulo \ref{sec:armazenamentoDados}). Esta versão antiga pode ser usada mais tarde, caso seja necessário recuperar informação, validar alterações, ou aplicar novamente a normalização dos dados. Esta funcionalidade não foi uma funcionalidade pensada para ser exposta diretamente ao utilizador, e não faz parte do uso normal, mas surgiu como consequência da arquitetura escolhida (ficheiros correntes e não correntes).

\subsubsection{Endpoints para comunicação com o frontend}

A aplicação expõe um conjunto de \textit{endpoints} \textit{rest} que permitem a interação com o frontend. Estes \textit{endpoints} são organizados de forma lógica, com rotas específicas para diferentes funcionalidades:

\begin{itemize}
    \item \texttt{quarters/new/}: Endpoint para criação de um novo \textit{quarter}.
    \item \texttt{quarters/edit/<uuid:uuid>/}: Permite editar os detalhes de um \textit{quarter} já existente, identificado pelo seu \gls{uuid}.
    \item \texttt{quarters/delete/<uuid:uuid>/}: Endpoint para apagar um \textit{quarter} específico.
    \item \texttt{quarters/}: Endpoint para listar todos os \textit{quarters} do utilizador.
    \item \texttt{quarters/files/delete/<uuid:uuid>/}: Endpoint para remover um ficheiro associado a um \textit{quarter}, usando o \gls{uuid} do ficheiro.
  \end{itemize}
  
A API inclui \textit{endpoints} para gestão de \textit{quarters} (\texttt{/api/\textit{quarters}/}) e visualização de gráficos (\texttt{/api/chart/})  O \textit{endpoint} para visualização de gráficos (\texttt{api/chart/}) é o mais importante, e é o que é utilizado para mostrar os gráficos na aplicação web, e permite uma série de parâmetros para filtrar os dados e mostrar gráficos de diferentes tipos. Estes endpoints são usados com recurso a pedidos HTTP (utilizando \texttt{fetch}) e também como páginas que aceitam parâmetros por \gls{url} (como por exemplo em formulários através do atributo \textttt{action}).

\begin{lstlisting}[language=\gls{html}, caption={Excerto do código \gls{html} do formulário de edição de \textit{quarter}}]
    <form method="post" action="/quarters/edit/b12fdf1f-8ce3-4050-a28f-07e444e15042/" id="edit-quarter-form" class="upload-form-wrapper" enctype="multipart/form-data">
      
      <div class="flex justify-end">
        <button type="submit" class="upload-modal-submit-btn">Save</button>
      </div>
    </form>
    \end{lstlisting}

Quando a aplicação carrega, não mostra logo todos os gráficos, mas sim um \textit{skeleton} que depois é substituido pelo gráfico, permitindo uma melhor experiência do utilizador. Esse carregamento é feito de forma assincrona, por Javascript, e utiliza o \textit{endpoint} para visualização de gráficos para obter a configuração final de um gráfico especifico a ser mostrado. A configuração retornada é especifica para a biblioteca \textit{Plotly}, que depois, juntamente com o restante código Javascript, consegue mostrar o gráfico e aplicar filtros com base nos parâmetros passados.

A comunicação com o \textit{backend} é feita através de chamadas assíncronas aos \textit{endpoints} desenvolvidos, passando parâmetros como o \textit{quarter} selecionado e os filtros ativos. O \textit{backend} devolve a estrutura necessária para renderização com \textit{Plotly}, assegurando que cada gráfico é gerado com os dados e configurações corretas.

\subsection{Interface Gráfica (Frontend)}

A interface da aplicação foi desenhada com foco na usabilidade e consistência visual, através da utilização de tecnologias modernas como \textit{WebComponents} e através de um \textit{design system}, Flowbite. O objetivo era garantir uma experiência responsiva, adequada para desktop e que suportasse minimamente dispositivos móveis, sem comprometer a performance nem a clareza na apresentação dos dados.

\subsubsection{Layout e Linguagem visual}

Um design system é uma biblioteca que oferece um conjunto de componentes reutilizáveis que podem ser usados em interfaces visuais. A escolha do Flowbite, que utiliza internamente a biblioteca Tailwind CSS, acelerou o desenvolvimento e simplificou a criação da interface visual, e de elementos como modais, formulários e botões. A coerência visual é reforçada por tipografia, cores e espaçamentos que já vem definidos por omissão na biblioteca.

A navegação é feita através de uma barra lateral, que mostra os gráficos disponíveis, enquanto modais são usados para operações críticas como confirmações e carregamento de ficheiros. Outras páginas são disponibilizadas numa barra de navegação no topo da página, que permite os utilizadores aceder à pagina de carregamento de ficheiros. Para o desenvolvimento do \gls{css} foi utilizada a linguagem SCSS, que permite desenvolver estilos de forma mais rápida e mais organizada, que depois é transpilada (através de um \textit{build system} como o ESBuild\cite{esbuild}) para \gls{css} normal.

Quando não existe gráficos a mostrar, é mostrada ao utilizador uma mensagem que indica que não existe ficheiros carregados, com uma hiperligação a redirecionar para a página de carregamento de ficheiros.

\subsubsection{WebComponents}

A camada de visualização de dados é composta por componentes Web, cada um encapsulando toda a sua a lógica, incluindo integrações com as bibliotecas \textit{Plotly} e \textit{Datatables} . Esta abordagem garante isolamento entre componentes, facilita a reutilização e simplifica a manutenção do código Javascript. 

A especificação \textit{WebComponents}\cite{webcomponents} permite criar os nossos nós \gls{dom}, e estende as interfaces \gls{dom} já existentes. Esta funcionalidade permite criar componentes que se comportam como elementos \gls{html}, e que podem ser utilizados como tal, e que também podem receber estilos e outras propriedades e atributos e é suportada já pelos os \textit{browsers} mais usados (como pode ser consultado no site \textit{Can I Use}\cite{caniuse} que é um site que indica o suporte de uma funcionalidade em diferentes \textit{browsers}).

\begin{figure}[H]
    \centering
    \includegraphics[max width=\textwidth]{./img/caniuse}
 \caption{Tabela de suporte - \textit{WebComponents}}
 \end{figure}

Na nossa aplicação, os \textit{WebComponents} são usados como elementos \gls{html} normais, que recebem apenas o gráfico que é para mostrar. Ao carregar, fazem uma chamada a um \textit{endpoint} do \textit{backend}, que retorna a configuração do gráfico, que é utilizada para mostrar o gráfico.

\begin{figure}[H]
    \centering
    \includegraphics[max width=\textwidth]{./img/webc}
 \caption{Utilização de \textit{WebComponents} na aplicação}
 \end{figure}

A comunicação com o \textit{backend} é feita através de chamadas assíncronas aos \textit{endpoints} desenvolvidos, passando parâmetros como o \textit{quarter} selecionado e os filtros ativos. O \textit{backend} devolve a configuração do gráfico para renderização com \textit{Plotly}, assegurando que cada gráfico é gerado com os dados e configurações corretas, como pode ser observado no diagrama de sequência \ref{fig:sequence}.

\begin{figure}[H]
    \centering
    \includegraphics[max width=\textwidth]{./img/sequence}
 \caption{Diagrama de sequência da comunicação com o \textit{backend}}
 \label{fig:sequence}
\end{figure}

A biblioteca \textit{Plotly} suporta uma configuração declarativa, ou seja, apenas precisamos de retornar um objeto estático, e através dessa configuração, a biblioteca consegue criar a visualização

\begin{lstlisting}[language=Javascript, caption={Excerto de uma configuração para um gráfico com a utilização da biblioteca \textit{Plotly}}]
  var trace1 = {
    x: ['giraffes', 'orangutans', 'monkeys'],
    y: [20, 14, 23],
    name: 'SF Zoo',
    type: 'bar'
  };
  
  var trace2 = {
    x: ['giraffes', 'orangutans', 'monkeys'],
    y: [12, 18, 29],
    name: 'LA Zoo',  
    type: 'bar'
  };
  
  var data = [trace1, trace2];
  var layout = {barmode: 'stack'};
  Plotly.newPlot('myDiv', data, layout);
\end{lstlisting}

Os gráficos são carregados de forma progressiva (\textit{lazy load}). Apenas são mostrados quando o nó entra dentro do \textit{viewport} do utilizador. O \textit{lazy load} permite que os servidor não fique sobre-carregado com pedidos, e que no lado do cliente, a memória utilizada seja minimizada visto que não são carregados todos os gráficos de uma vez.	

As bibliotecas \textit{Plotly} e \textit{Datatables} foram integradas diretamente com os \textit{WebComponents}\cite{webcomponents}, fazendo com que o WebComponent consiga mostrar gráficos e tabelas de forma encapsulada. Esta integração permite que, quando o carregamento de um gráfico é iniciado, que seja apresentado  um \textit{skeleton} (uma representação visual com um \textit{spinner} e com a largura e altura aproximada de um gráfico) que é trocada pelo o gráfico em si quando os dados ficam disponíveis. A ideia deste \textit{skeleton} é que o utilizador veja que o gráfico está a ser carregado, e que não cause um grande deslocamento do gráfico na interface (\textit{layout shifting}) quando os dados estão prontos.

\subsubsection{Responsividade em dispositivos móveis}

Toda a interface foi pensada para ser utilizada em computadores devido à quantidade de gráficos que são apresentados. A interface adapta-se bem a dispositivos moveis, mas devido ao espaço disponivel, não é possível garantir uma boa experiencia de utilização em dispositivos moveis. A aplicação tem, essencialmente, duas \textit{media queries}. Uma \textit{media query} que até aos 1023px que inclui telemóveis, tablets e ecrãs pequenos, e uma \textit{media query} a partir dos 1024px que inclui computadores e ecrãs maiores.

\begin{figure}[H]
    \centering
    \includegraphics[max max width=\textwidth]{./img/res_1023}
 \caption{Media query até 1023px}
\end{figure}

\begin{figure}[H]
    \centering
    \includegraphics[max max width=\textwidth]{./img/res_1920}
 \caption{Media query a partir dos 1024px}
\end{figure}

Estes \textit{media queries} são as mais comuns, garantindo uma boa experiência de utilização em todos os ecrãs, sendo que não tivemos de fazer código explicitamente para a aplicação se adaptar para dispositivos móveis. Os filtros permitem alterar os gráficos em tempo real, com \textit{feedback} visual imediato não sendo necessário recarregar a página. No momento em que o filtro troca, é feito um pedido a API de gráficos que retorna um novo conjunto de dados para mostrar, com a configuração do gráfico atualizada.

\subsubsection{Carregamento de ficheiros}

A organização dos dados foi pensada para refletir a lógica do projeto Marketplace Simulations. Cada utilizador pode criar múltiplos \textit{quarters} no momento em que carrega um ficheiro. Estes \textit{quarters} funcionam como buckets isolados onde são armazenados os ficheiros carregados. A navegação entre \textit{quarters} é facilitada por setas laterais em cada gráfico apresentado. A criação e remoção de \textit{quarters}, bem como o carregamento de ficheiros, é acompanhada por feedback visual no momento da ação.

Para verificar se os ficheiros são do tipo correto, decidimos validar através do \textit{MIME type}, que é um identificador  utilizado para descrever o tipo de conteúdo de um ficheiro. Os \textit{MIME types} seguem o formato \texttt{tipo/subtipo}, como por exemplo \texttt{text/csv} para ficheiros CSV ou \texttt{application/\allowbreak vnd\allowbreak.open\allowbreak xmlV formats-\allowbreak officedocument.\allowbreak spreadsheetml.\allowbreak sheet} para ficheiros Excel no formato \texttt{.xlsx}. Esta abordagem permite garantir que os ficheiros carregados correspondem realmente ao formato esperado, independentemente da extensão do nome do ficheiro que pode ser facilmente manipulada. 

Ao validar o \textit{MIME type} do ficheiro, conseguimos rejeitar ficheiros com o tipo errado. No contexto deste projeto, esta verificação era essencial, uma vez que apenas ficheiros \texttt{.xlsx} válidos devem ser processados. Assim, a validação por \textit{MIME type} contribui tanto para a segurança como para a integridade da aplicação.

Durante o carregamento, a interface valida se os ficheiros são do tipo correto. Apenas ficheiros dos mime type \texttt{application/vnd\allowbreak.openxmlformats\allowbreak officedocument\allowbreak.spreadsheetml\allowbreak.sheet} e \texttt{application/vnd.ms\allowbreak excel} são permitidos.

%%________________________________________________________________________
\chapter{Validação e Testes}
\label{ch:validacaoTestes}
%%________________________________________________________________________

A validação e testes da aplicação foram realizados através de múltiplas abordagens, garantindo uma cobertura de testes  da funcionalidade como da usabilidade da aplicação. Este capítulo descreve as diferentes estratégias de teste implementadas e os resultados obtidos.

\section{Testes de Acessibilidade}

A acessibilidade foi uma preocupação no desenvolvimento da aplicação, e para isso foram seguidas as diretrizes \gls{wcag} 2.1, que definem critérios técnicos para tornar os conteúdos web mais acessíveis a todos os utilizadores, incluindo pessoas com deficiências visuais, motoras, auditivas, cognitivas ou neurológicas. Para garantir a conformidade com estas diretrizes, recorremos a uma abordagem prática combinada de testes automáticos e validação manual. Entre as ferramentas utilizadas destacam-se duas principais: o Lighthouse e o Axe.

O Lighthouse, desenvolvido pela Google, é uma ferramenta open source que corre diretamente no Chrome DevTools ou como \gls{cli}. Permite auditar uma página web em várias categorias, nomeadamente Performance, SEO, Best Practices e, claro, Accessibility. A secção de acessibilidade do Lighthouse verifica, por exemplo, se os elementos têm contrastes suficientes, se os formulários estão corretamente identificadas com atributos aria-label, se há títulos estruturados (h1, h2, etc.) numa hierarquia lógica e se os elementos interativos (como botões e links) são acessíveis por teclado. O Lighthouse fornece uma pontuação de 0 a 100 e destaca problemas comuns com sugestões de como resolver.

Esta ferramenta Lighthouse, apesar de não ser tão detalhada como a Axe, sinalizou alguns problemas, como a falta de compressão \gls{gzip} e alguns erros de acessibilidade e \gls{seo}. Estes problemas foram resolvidos (fazendo alterações na configuração do servidor Nginx e no código da aplicação), e a ferramenta Lighthouse passou a dar uma pontuação melhor nos vários critérios que avalia.

\begin{figure}[H]
\centering
\includegraphics[max width=\textwidth]{./img/lh_before}
\caption{Testes de acessibilidade com a ferramenta Lighthouse}
\end{figure}

TODO: Adicionar imagem do Lighthouse após as correções.
\begin{figure}[H]
\centering
\includegraphics[max width=\textwidth]{./img/lh_before}
\caption{Testes de acessibilidade com a ferramenta Lighthouse após as correções}
\end{figure}

Já o Axe, da empresa Deque, é uma biblioteca de testes de acessibilidade mais focada, muito utilizada em ambientes de desenvolvimento profissional. Pode ser usada como extensão do browser ou integrada com ferramentas de testes automatizados como Cypress ou Selenium. O Axe segue os critérios \gls{wcag} com mais profundidade e reporta violações como a ausência de nomes acessíveis (\textit{accessible names}), uso incorreto de landmarks \gls{aria}, falhas de foco, ou elementos visuais sem correspondência textual. Além disso, o Axe permite navegar entre os erros diretamente na interface do browser, o que torna o processo de debugging muito mais fácil.

Nos primeiros testes feitos com a ferramenta Axe, as ferramentas sinalizaram alguns problemas relativos a acessibilidade, como a falta de textos alternativos para alguns elementos visuais, e os niveis dos titulos não estavam a ser usados de forma adequada.

\begin{figure}[H]
    \centering
    \includegraphics[max width=\textwidth]{./img/axe}
    \caption{Testes de acessibilidade com a ferramenta Axe}
    \end{figure}


Após resolver os problemas indicados pela ferramenta Axe, conseguimos então melhorar a acessibilidade da aplicação, como se pode ver na imagem abaixo.

\begin{figure}[H]
\centering
\includegraphics[max width=\textwidth]{./img/axe_after}
\caption{Testes de acessibilidade com a ferramenta Axe após as correções}
\end{figure}

No nosso caso, ambas as ferramentas foram utilizadas de forma complementar: o Lighthouse ajudou-nos a identificar problemas mais genéricos e dar uma visão rápida do estado da acessibilidade da aplicação, enquanto o Axe foi essencial para validar critérios mais técnicos e detetar problemas complexos que o Lighthouse pode não cobrir. Os testes foram feitos tanto em páginas com dados carregados como em estados vazios, e procurámos garantir que todas as interações essenciais da plataforma,  como o Carregamento de ficheiros, usabilidade e navegação entre gráficos fossem utilizáveis por utilizadores com leitores de ecrã e navegação apenas por teclado.

Desta forma, procurámos alinhar a aplicação com boas práticas de desenvolvimento inclusivo, respeitando não só normas técnicas mas também a responsabilidade social de criar interfaces acessíveis a todos, com foco nas seguintes funcionalidades:
\begin{itemize}
    \item Adição de textos alternativos para todos os elementos visuais
    \item Garantia de contraste adequado entre texto e fundo
    \item Implementação de navegação por teclado
    \item Estruturação semântica de \gls{html}
    \item Suporte a leitores de ecrã
\end{itemize}

\section{Testes Manuais}

Para complementar os testes das ferramentas acima, foi desenvolvido um conjunto abrangente de testes manuais utilizando a linguagem Gherkin para descrever os cenários de teste. Gherkin é uma linguagem estruturada, mas legível por humanos, que permite descrever comportamentos esperados do sistema em forma de cenários do tipo Dado-Quando-Então. Esta abordagem facilita a comunicação entre equipas técnicas e não técnicas, garantindo que todos os intervenientes compreendam os objetivos de cada teste.

Foi pedido a cinco testadores independentes para executarem estes testes manualmente, com base nos cenários escritos em Gherkin, e no final preencherem um formulário com os resultados de cada execução, incluindo observações e eventuais desvios face ao comportamento esperado. Os cenários de teste foram organizados em duas categorias principais:

\begin{itemize}
    \item Fluxos de utilizador (criação de conta, login, navegação)
    \item Operações com dados (carregar e apagar ficheiros, visualização de gráficos)
\end{itemize}

Cada cenário foi documentado usando a sintaxe Gherkin, como exemplificado abaixo. Os restantes cenários Gherkin foram incluídos no apêndice \ref{ch:cenariosGherkin}.

\begin{lstlisting}[language=Gherkin, caption={Excerto do código Gherkin do cenário de teste para a criação de uma conta}]
Scenario: Creating an account
	Given I access the page 
	And I don't have an account or logged in
	Then I should see the "Create Account" link
	When I click the "Create Account" link
	Then I should be redirected to the "Create Account" form
	And I should see the username field
	And I should see the password filed
	And I should see the confirm password field
	When I fill that form
	And I click "Save"
	Then I should be redirected to the "Login" page
\end{lstlisting}

Apesar desta amostra não ser muito representativa, foi possivel encontrar bugs na aplicação, como por exemplo, a falta de alguns dados em alguns gráficos, e alguns gráficos náo serem coerentes com a informação que estavam a mostrar. Estes bugs foram corrigidos assim que foram reportados pelos testadores.

TODO: Falta falar sobre a análise dos resultados dos testes manuais.

\section{Compatibilidade com Navegadores}

A compatibilidade com diferentes \textit{browsers} foi testada utilizando a plataforma BrowserStack, que permitiu testar a aplicação em múltiplos ambientes. Os testes foram realizados nas versões mais recentes dos principais \textit{browsers}:

\begin{itemize}
    \item Google Chrome (versões acima da 90)
    \item Mozilla Firefox (versões acima da 88)
    \item Microsoft Edge (versões acima da 90)
    \item Safari (versões acima da 18)
\end{itemize}

A aplicação é compatível com os \textit{browsers} testados, nas versões em que validámos o seu funcionamento.

%%_______________________________________________________________________
\chapter{Conclusões e Trabalho Futuro}
\label{ch:conclusoesTrabalhoFuturo}
%%________________________________________________________________________

O desenvolvimento deste projeto culminou numa aplicação web funcional que responde de forma direta e eficaz às necessidades dos estudantes do ISCAL no que diz respeito à análise dos dados exportados da plataforma Marketplace Simulations. Desde o início, o objetivo foi construir uma solução prática, que facilitasse a leitura e exploração da informação, e ao mesmo tempo fosse suficientemente robusta e escalável para crescer com os utilizadores e os dados.

\section{Conclusões}

Um dos pontos fortes deste projeto foi a implementação de um sistema automático para gerir ficheiros e transformar dados em gráficos. Esta funcionalidade, de os utilizadores poderem carregar ficheiros diretamente exportados do simulador e, a partir daí, obterem gráficos interativos, sem qualquer necessidade de formatação manual representa uma melhoria relação ao processo anterior, que era manual, demorado e propenso a erros. 

A decisão de usar Django como base para o \textit{backend} revelou-se acertada. Além de facilitar a autenticação e gestão de utilizadores, a  biblioteca permitiu implementar uma boa arquitetura, com separação entre as camadas frontend e \textit{backend}. A autenticação integrada, associada ao isolamento de dados por utilizador, garantiu uma camada extra de segurança sem complicar a lógica da aplicação. A API REST que foi construída permitiu uma comunicação fluida entre as várias partes do sistema, suportando atualizações dinâmicas dos gráficos sem impacto na performance.

No lado do frontend, optou-se por uma abordagem prática, usando bibliotecas para garantir a responsabilidade e usabilidade. A biblioteca \textit{Plotly} foi fundamental para criar gráficos interativos que, além de apelativos, ajudam os estudantes a interpretar os dados.

Durante o desenvolvimento surgiram alguns desafios técnicos. Um dos principais foi lidar com a diversidade estrutural dos ficheiros. Alguns traziam folhas com formatação irregular, nomes de colunas inconsistentes ou dados dificeis de perceber. Isso obrigou à criação de uma pipeline de normalização robusto e reutilizável que é uma componente essencial para garantir que os dados fossem processados corretamente e de forma previsível. Apesar da solução atual funcionar bem no contexto do projeto, poderá ser necessário alterar no futuro caso os ficheiros evoluam ou incluam novos tipos de dados. Outro ponto que pode ser melhorado é a interface de gestão de \textit{quarters}, que embora funcional, ainda exige alguma familiaridade com o conceito por parte do utilizador.

Para além do lado técnico, este projeto foi também uma excelente oportunidade de crescimento pessoal. Antes de começar, não tinha experiência prática com Django nem com bibliotecas como Pandas, e o contacto com estas tecnologias acabou por ser fundamental. Esta aprendizagem provou o valor da adaptação contínua e da experimentação prática como pilares fundamentais do desenvolvimento de software.

Em síntese, este projeto mostrou como a combinação de ferramentas modernas pode resultar numa solução escalável e com impacto direto na experiência de aprendizagem. A aplicação desenvolvida resolve um problema enfrentado pelos estudantes do ISCAL, e deixa aberta a porta para futuras melhorias e novos casos de uso. É uma base sólida, que pode ser facilmente estendida ou adaptada a outros contextos educacionais onde a análise de dados seja um desafio.

\section{Trabalho Futuro}

Apesar de o sistema atual já cumprir os objetivos definidos inicialmente, foram identificadas diversas oportunidades de evolução e expansão que poderão ser exploradas em futuras iterações do projeto. Estas propostas estão organizadas em diferentes áreas de desenvolvimento.

\subsection{Interface e Experiência do Utilizador}

Uma das direções mais imediatas para evolução está ao nível da interface. Pretende-se alargar o leque de visualizações disponíveis, incluindo novos tipos de gráficos, como dispersões, histogramas ou até mapas interativos, sendo que é sempre importante ter em mente a experiencia de utilização. Para além disso, seria interessante permitir a criação de \textit{dashboards} personalizáveis, onde cada utilizador poderia compor visualizações de acordo com os seus objetivos, e até juntar vários tipos de dados na mesma visualização.

Outro aspeto relevante prende-se com a configuração dinâmica dos gráficos. A ideia é deixar de apresentar visualizações estáticas e passar a oferecer ao utilizador a possibilidade de escolher, em tempo real, os tipos de gráfico, métricas e dimensões que pretende analisar. Esta abordagem aumentaria substancialmente a flexibilidade da ferramenta.

\subsection{Análise de Dados e Otimização de Performance}

Do ponto de vista analítico, uma evolução natural do sistema passa pela incorporação de técnicas de análise preditiva. A utilização de modelos de regressão ou algoritmos de aprendizagem automática simples poderá ajudar os estudantes a identificar padrões e antecipar tendências de mercado com base nos dados da simulação.

Em paralelo, importa melhorar a eficiência do sistema no processamento de dados de maior volume. Será relevante avaliar o uso de bibliotecas como \texttt{Dask} ou \texttt{Polars}, que oferecem soluções otimizadas para manipulação de grandes \textit{datasets}, mantendo tempos de resposta aceitáveis e garantindo uma boa experiência de utilização.

\subsection{Gestão de Utilizadores e Colaboração}

Ao nível da gestão da plataforma, propõe-se o reforço das funcionalidades de administração de utilizadores. Isto poderá incluir a criação de perfis diferenciados (por exemplo, administrador, editor, visualizador) e mecanismos de controlo de permissões.

Adicionalmente, seria interessante introduzir funcionalidades de colaboração entre utilizadores, nomeadamente a partilha de visualizações e filtros de dados. Este tipo de funcionalidade é particularmente útil em contextos académicos, onde os trabalhos são frequentemente realizados em grupo ao longo do semestre.

\subsection{Integração com Outras Plataformas}

Por fim, um caminho a explorar passa pela integração direta com a plataforma \textit{Marketplace Simulations}, de onde os dados são originalmente extraídos. Uma abordagem possível seria o desenvolvimento de extensões de \textit{browser} (para \textit{Chrome}, \textit{Firefox} ou \textit{Edge}) que permitam importar automaticamente os dados para o nosso sistema, eliminando etapas manuais do processo de upload.

De forma geral, estas propostas de melhoria mostram que o projeto pode continuar a evoluir significativamente, servindo como base para novas investigações ou aplicações.
%%________________________________________________________________________


\appendix

%%________________________________________________________________________
%%________________________________________________________________________
%% LEIM | PROJETO
%% 2022 / 2013 / 2012
%% Modelo para relatório
%% v04: alteração ADEETC para DEETC; outros ajustes
%% v03: correção de gralhas
%% v02: inclui anexo sobre utilização sistema controlo de versões
%% v01: original
%% PTS / MAR.2022 / MAI.2013 / 23.MAI.2012 (construído)
%%________________________________________________________________________
\chapter{Tabelas de Requisitos Funcionais e Não Funcionais}
\label{ch:tabRequisitos}

\textbf{Requisitos Funcionais}

\begin{itemize}
\item \textbf{Visíveis}: São funções obrigatórias que devem estar presentes e ser visíveis para o utilizador. Representam ações diretamente acessíveis ou observáveis na interface da aplicação.

\item \textbf{Invisíveis}: Também obrigatórias, mas não visíveis para o utilizador final. Estas funções dizem respeito a comportamentos internos do sistema, como validações, persistência de dados ou processamento em segundo plano.
\end{itemize}

Os requisitos estão ordenados de acordo com a sua prioridade.

\begin{table}[H]
\centering
\begin{tabular}{|l|p{7cm}|l|}
\hline
\textbf{Requisito} & \textbf{Função} & \textbf{Categoria} \\
\hline
R1.1 & Permitir criação de conta - utilizador único & Visível \\
R1.2 & Permitir login & Visível \\
\hline
\end{tabular}
\caption{Requisitos de Autenticação}
\label{tab:requisitosAutenticacao}
\end{table}

\begin{table}[H]
\centering
\begin{tabular}{|l|p{7cm}|l|}
\hline
\textbf{Requisito} & \textbf{Função} & \textbf{Categoria} \\
\hline
R2.1 & Permitir upload de ficheiros & Visível \\
R2.2 & Associar ficheiros carregados a trimestres & Invisível \\
\hline
\end{tabular}
\caption{Requisitos de Gestão de Ficheiros}
\label{tab:requisitosFicheiros}
\end{table}

\begin{table}[H]
\centering
\begin{tabular}{|l|p{7cm}|l|}
\hline
\textbf{Requisito} & \textbf{Função} & \textbf{Categoria} \\
\hline
R3.1 & Permitir eliminação de ficheiros carregados & Visível \\
R3.2 & Permitir criação de trimestres identificados por ``\textit{Quarter N}'' & Visível \\
R3.3 & Listar todos os trimestres do utilizador & Visível \\
\hline
\end{tabular}
\caption{Requisitos de Gestão de Trimestres}
\label{tab:requisitosTrimestres}
\end{table}

\begin{table}[H]
\centering
\begin{tabular}{|l|p{7cm}|l|}
\hline
\textbf{Requisito} & \textbf{Função} & \textbf{Categoria} \\
\hline
R4.1 & Visualizar gráficos & Visível \\
R4.2 & Aplicar filtros & Visível \\
\hline
\end{tabular}
\caption{Requisitos de Visualização de Dados}
\label{tab:requisitosVisualizacao}
\end{table}

\textbf{Requisitos Não-Funcionais}
Os atributos do sistema, também designados por requisitos não funcionais, representam as características que a aplicação deve exibir durante a sua utilização. Tal como os requisitos funcionais, estes estão organizados em categorias distintas, de forma a informar sobre o seu impacto no desenvolvimento da aplicação. As principais categorias consideradas são as seguintes:

\begin{itemize}
\item \textbf{Obrigatórios}: a aplicação deve cumprir estes requisitos. Estão geralmente associados a restrições técnicas ou de contexto, como desempenho, segurança, compatibilidade entre plataformas ou integridade dos dados.

\item \textbf{Desejáveis} — São requisitos que, embora não sejam críticos para o funcionamento do sistema, representam funcionalidades adicionais ou melhorias que poderão ser implementadas numa fase posterior. a aplicação deve estar preparado para os acomodar, caso se decida avançar com a sua implementação.
\end{itemize}

Os requisitos estão ordenados de acordo com a sua prioridade.
\begin{table}[H]
    \centering
    \begin{tabular}{|l|p{7cm}|l|}
    \hline
    \textbf{Atributo} & \textbf{Detalhe / Restrição - Fronteira} & \textbf{Categoria} \\
    \hline
    Usabilidade & Detalhe - Interface intuitiva & Obrigatório \\
    Usabilidade & Detalhe - Carregamento de gráficos em sem bloquear o utilizador (\textit{lazy load}) & Obrigatório \\
    Usabilidade & Detalhe - Suporte para múltiplos \textit{browsers} & Obrigatório \\
    Segurança & Detalhe - Autenticação e Contas & Obrigatório \\
    Segurança & Fronteira - Cada utilizador só pode aceder aos seus dados & Obrigatório \\
    Segurança & Detalhe - Garantir que a aplicação só permite ficheiros com formato previsto & Obrigatório \\
    Performance & Detalhe - Suporte para múltiplos utilizadores sem degradação significativa & Desejável \\
    Performance & Detalhe - Resposta rápida às interações do utilizador & Obrigatório \\
    Acessibilidade & Detalhe - Tem de ser navegável por teclado e \textit{screen-reader} \textit{friendly} & Obrigatório \\
    Dados & Normalizar os dados que recebe de forma a serem apresentáveis & Obrigatório \\

    \hline
    \end{tabular}
    \caption{Tabela de Requisitos Não Funcionais}
    \label{tab:requisitosNaofuncionais}
    \end{table}

\chapter{Comparação de gráficos suportados pelo \textit{Plotly} e \textit{Chart.js}}
\label{ch:charts}

\begin{table}[H]
\centering
\caption{Comparação de tipos de gráficos suportados por \textit{Plotly}.js e \textit{Chart.js}}
\begin{tabular}{|l|c|c|}
\hline
\textbf{Tipo de Gráfico} & \textbf{\textit{Plotly.js}} & \textbf{\textit{Chart.js}} \\
\hline
Barras                         & Suportado & Suportado \\
Linhas                         & Suportado & Suportado \\
Dispersão (Scatter)            & Suportado & Suportado \\
Circular (Pie)                 & Suportado & Suportado \\
Área                           & Suportado & Suportado \\
Radar                          & Suportado & Suportado \\
Histograma                     & Suportado & Suportado \\
Box Plot                       & Suportado & Não Suportado \\
Mapa de Calor (Heatmap)        & Suportado & Não Suportado \\
Cascata (Waterfall)            & Suportado & Não Suportado \\
Indicadores (Gauges)           & Suportado & Não Suportado \\
Candlestick (Finanças)         & Suportado & Não Suportado \\
OHLC (Finanças)                & Suportado & Não Suportado \\
Treemap                        & Suportado & Não Suportado \\
Sunburst                       & Suportado & Não Suportado \\
Violin                         & Suportado & Não Suportado \\
Mapa (Geo)                     & Suportado & Não Suportado \\
\hline
\end{tabular}
\label{tab:charts}
\end{table}

\chapter{Controlo de Versões e GitHub}

O controlo de versões é essencial em projetos de de desenvolvimento de \textit{software}, uma vez que permite acompanhar alterações ao código, colaborar com várias pessoas e garantir a consistência do código ao longo do tempo. No contexto deste projeto, usamos controlo de versões durante a implementação da plataforma, para permitir uma evolução controlada do código.

A aplicação de controlo de versões que decidimos usar foi o \textit{Git}, que é uma tecnologia utilizada no setor e uma das mais antigas e mais bem suportadas. O \textit{Git} permite criar \textit{branches}, o que facilita o desenvolvimento em paralelo de novas funcionalidades sem interferir com a versão principal da aplicação.

Durante o desenvolvimento da plataforma, esta capacidade de criar \textit{branches} foi útil para testar funcionalidades específicas como o processamento dos ficheiros, integrar a gestão de utilizadores no código já existente, e para desenvolver a funcionalidade de carregamento e normalização dos dados. Cada conjunto de funcionalidades foram desenvolvidas de forma isolada, permitindo testes controlados e evitando conflitos no código principal. Uma \textit{branch} é uma cópia isolada do código de um repositório, onde podemos desenvolver novas funcionalidades, corrigir erros ou testar alterações sem afetar a versão principal (geralmente chamada de \textit{master}).

Além do \textit{Git}, foi utilizado o \textit{GitHub}. O \textit{GitHub} é uma serviço \textit{cloud} que permite partilhar repositórios \textit{Git}, e estende o processo de colaboração com a funcionalidade como o acompanhamento de tarefas e \textit{issues}. No desenvolvimento do projeto utilizámos \textit{branches} para conseguir desenvolver as funcionalidades de forma isolada. 

Após cada \textit{commit} para o \textit{branch} principal é corrido um \textit{workflow} que automaticamente atualiza a versão do código que corre dentro da \gls{vps} que usamos para fazer o correr a aplicação (o nosso ambiente de produção). Este \textit{workflow} liga-se por \gls{ssh} ao servidor, puxa a última versão do \textit{branch} principal, e reinicia os \textit{containers} da aplicação, aplicando as alterações aos modelos de dados e aos ficheiros estáticos.

\chapter{Cenários \textit{Gherkin}}
\label{ch:cenariosGherkin}

Em seguida, apresentam-se os cenários \textit{Gherkin} que foram utilizados para testar a aplicação.

\begin{lstlisting}[language=Gherkin]
Scenario: Creating an account
    Given I access the page 
    And I don't have an account or logged in
    Then I should see the "Create Account" link
    When I click the "Create Account" link
    Then I should be redirected to the "Create Account" form
    And I should see the username field
    And I should see the password filed
    And I should see the confirm password field
    When I fill that form
    And I click "Save"
    Then I should be redirected to the "Login" page
\end{lstlisting}



\begin{lstlisting}[language=Gherkin]
Background:
	Given that I have an account

Scenario: Login on the app
	When access the \textit{web}site
	And I am not logged in
	Then I should see the Login page
	And I should see the username field
	And I should see the password filed
	When I fill that form with my login details
	And I click "Save"
	Then I should be redirected to the "Home" page
\end{lstlisting}
    
\begin{lstlisting}[language=Gherkin]
Background:
	Given that I have an account
	And I am logged in on the application

Scenario: Uploading Q1 files
	When I access the "Uploads" page  
	And I click on "Upload files" button 
	Then the upload modal appears 
	When i click on the upload files box
	And I upload all files from Q1 folder
	And I click "Save"
	Then I should see the files that i just uploaded on the page
	When I navigate to the "Home" page
	Then I should see charts. 
	And I should be able to zoom on the chart
	
Scenario: Uploading Q2 files
	When I access the "Uploads" page  
	And I click on "Upload files" button 
	Then the upload modal appears 
	When i click on the upload files box
	And I upload all files from Q2 folder
	And I confirm that the "Quarter Number" has the value of "2".
	And I click "Save"
	Then I should see the files that i just uploaded on the page
	When I navigate to the "Home" page
	Then I should see charts
	And on the "Customer  Needs" chart, i should be able to navigate to Q1.
	And I should be able to zoom on the chart
\end{lstlisting}

\begin{lstlisting}[language=Gherkin]
    Background:  
	Given that I have an account
	And I am logged in on the application
	And I have uploaded files to the platform
 
Scenario: Removing a file
	When I navigate to the "Uploads" page  
	Then I should see the files that i already uploaded
	When I click on X button on the file "Customer Needs"
	Then the file is removed from the quarter
	And I should not see charts from that file
	
Scenario: Removing a quarter
	When I navigate to the "Uploads" page  
	Then I should see the files that i already uploaded
	And I should see a "Delete" button.
	When I click on the "Delete" button
	Then I should see all of the files from that section removed 
\end{lstlisting}


%%________________________________________________________________________

\bibliographystyle{plain}
\backmatter
\bibliography{Xbib}


\end{document}
