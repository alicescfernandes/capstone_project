%%________________________________________________________________________
%% LEIM | PROJETO
%% 2022 / 2013 / 2012
%% Modelo para relatório
%% v04: alteração ADEETC para DEETC; outros ajustes
%% v03: correção de gralhas
%% v02: inclui anexo sobre utilização sistema controlo de versões
%% v01: original
%% PTS / MAR.2022 / MAI.2013 / 23.MAI.2012 (construído)
%%________________________________________________________________________




%%________________________________________________________________________
\chapter{Sistemas de Apoio à Decisão}
\label{ch:sad}
%%________________________________________________________________________

Um \gls{sad} é uma aplicação ou conjunto de ferramentas desenhadas para ajudar os utilizadores a tomar decisões mais informadas e fundamentadas, geralmente com base na análise de dados. Na prática, um \gls{sad} recolhe, organiza e processa dados, muitas vezes em tempo real ou a partir de ficheiros carregados pelo utilizador, e apresenta esses dados de forma visual (como gráficos ou dashboards), permitindo aplicar filtros e explorar cenários. Ou seja, não toma decisões por si só, mas fornece os elementos certos para que o utilizador possa decidir melhor, especialmente em contextos complexos ou com grande volume de informação.


%%________________________________________________________________________
\chapter{Casos de Utilização - UML}
\label{ch:casosUtilizacaoUml}
%%________________________________________________________________________

\chapter{Tabela de Requisitos funcionais}
\label{ch:outroDetalheAdicional}


\chapter{Tabela de Requisitos não funcionais}
\label{ch:outroDetalheAdicional}


\chapter{Classificação das folhas \gls{xlsx}}
\label{ch:outroDetalheAdicional}









