%%________________________________________________________________________
%% LEIM | PROJETO
%% 2022 / 2013 / 2012
%% Modelo para relatório
%% v04: alteração ADEETC para DEETC; outros ajustes
%% v03: correção de gralhas
%% v02: inclui anexo sobre utilização sistema controlo de versões
%% v01: original
%% PTS / MAR.2022 / MAI.2013 / 23.MAI.2012 (construído)
%%________________________________________________________________________


%%________________________________________________________________________
\chapter{Introdução}
\label{ch:introducao}
%%________________________________________________________________________

No contexto do ensino superior, tem-se dado cada vez mais importância à integração de ferramentas tecnológicas que tornem o processo de aprendizagem mais eficaz e prático, como é o caso do simulador utilizado pelos alunos do \gls{iscal}. Esta ferramenta permite que os estudantes apliquem, de forma interativa, os conhecimentos adquiridos ao longo do curso, colocando-os em situações de tomada de decisão semelhantes às que enfrentariam num ambiente real. Deste modo, o simulador não só reforça os conteúdos teóricos, como também promove o desenvolvimento de competências analíticas e estratégicas, essenciais para a sua formação. 

Neste contexto, a análise eficiente de dados (da plataforma de simulação) torna-se essencial para a tomada de decisões estratégicas e tem impacto na avaliação final dos alunos, no entanto, a complexidade e a falta de visuais que ajudem a perceber as informações apresentadas pela plataforma original podem representar um desafio significativo para os estudantes.

\section{Motivação}
O presente projeto surgiu da necessidade  entre os alunos do \gls{iscal} que utilizam o simulador \textit{Marketplace Simulations - International Corporate Management} (que iremos descrever no capítulo seguinte), que é um simulador de negócios internacionais, onde os alunos se agrupam em empresas e simulam a criação de um negócio num mercado internacional. Embora a plataforma forneça toda a informação necessária à tomada de decisões na simulação, esses dados estão dispersos em múltiplas secções e apresentados em tabelas, com poucas funcionalidades de visualização gráfica. Esta limitação obriga os alunos a alternar entre páginas, copiar dados manualmente e criar folhas de cálculo externas, comprometendo a eficiência e a análise da informação.


\section{Objetivos}

A aplicação proposta neste relatório pretende ajudar nesse sentido, oferecendo uma interface que permite aos utilizadores carregar dados retirados da plataforma de simulação e tornar esses ficheiros em visualizações que podem ser consultadas e manipuladas. A aplicação permitirá aos utilizadores:
\begin{itemize}
    \item Criar uma conta na plataforma que permita persistir a informação carregada.
    \item Carregar ficheiros exportados da plataforma original;
    \item Visualizar os dados em gráficos interativos;
\end{itemize}

Do ponto de vista técnico, queremos que  projeto adote uma arquitetura fácil de manter e que vá de encontro à utilização da plataforma, dando ênfase aos seguintes itens:
\begin{itemize}
    \item A normalização e transformação automática de dados provenientes de fontes externas;
    \item A facilidade na gestão de ficheiros, com o objetivo de oferecer uma interface facil para os utilizadores finais.
    \item Organizar a informação por utilizador, garantindo que o utilizador apenas consegue consultar a informação carregada.
    \item Adotar um modelo de funcionamento semelhante à plataforma de simulação, de modo a tornar a experiência de utilização mais intuitiva e garantido que a nossa aplicação tenha fronteiras claras de utilização.
\end{itemize}

Ao longo deste relatório, serão detalhadamente apresentadas as decisões tomadas, bem como os fundamentos que orientaram o desenvolvimento da aplicação proposta.

%%________________________________________________________________________
\chapter{Trabalho Relacionado}
\label{ch:trabalhoRelacionado}
%%________________________________________________________________________

O presente projeto insere-se num contexto mais geral de ferramentas pedagógicas e \gls{sad}, ainda que neste caso concreto,  em ambientes simulados no ensino superior. No âmbito do \gls{iscal}, é frequente a utilização de plataformas como a que iremos descrever nos seguintes capítulos \textit{Marketplace Simulations}, que permitem aos estudantes desenvolver competências práticas em ambientes virtuais de negócios, replicando o funcionamento de mercados reais. Esta necessidade de suporte digital à análise e simulação motivou o desenvolvimento de outras ferramentas auxiliares, com destaque para um projeto também realizado em parceria com o \gls{iscal}, focado em simulações parciais de modelos económicos.

Esse projeto, embora partilhe uma motivação semelhante (criar micro-simulações) apresenta uma abordagem distinta. Em particular, a aplicação permite simular cenários específicos com base em inputs manuais, o que pode ser útil para quando se procura prever resultados com foco muito concreto (por exemplo, simular o impacto de uma única variável nos resultados). Esta plataforma foca-se numa outra vertente, em que pretende ser uma ferramenta para auxiliar o ensino de forma prática.

O projeto atual distingue-se, tanto pelo seu foco como pela componente técnica. Ao contrário do sistema anterior, que se baseava em simulações parametrizadas modelos estáticos, esta nova aplicação aposta numa abordagem orientada a dados, onde a informação real dos jogos de simulação é carregada diretamente pelo utilizador. Essa escolha trouxe consigo a necessidade de resolver questões de normalização, estruturação e tratamento de dados, algo que não era pedido no projeto relacionado.

Este trabalho assume, portanto como pressuposto, o uso de dados extraídos diretamente da plataforma de simulação, e a valorização da experiência do utilização na apresentação de dados. Em conjunto, estes elementos definem uma solução mais abrangente e alinhada com os desafios técnicos reais da análise de dados em contexto educativo, indo além das simulações reduzidas ou controladas que caracterizavam os trabalhos relacionados anteriormente desenvolvidos.

%%________________________________________________________________________
\chapter{Modelo Proposto}
\label{ch:modeloProposto}
%%________________________________________________________________________

(falta introduzir o modelo proposto)


%%________________________________________________________________________
\section{Requisitos}
\label{sec:requisitos}
%%________________________________________________________________________

\subsubsection{Requisitos funcionais}
Os requisitos funcionais descrevem as funcionalidades específicas que a aplicação deve oferecer para atender às necessidades dos utilizadores. No contexto deste projeto, definem as ações que o sistema deve ser capaz de executar. No nosso projeto, identificamos os seguintes requisitos, ordenados por prioridade:

\begin{itemize}
    \item \textbf{Visualização de dados:} os utilizadores devem poder visualizar gráficos interativos baseados nos dados carregados, com a possibilidade de aplicar filtros como país ou quarter selecionado.

    \item \textbf{Gestão de ficheiros:} os utilizadores devem poder carregar ficheiros, associá-los a quarters e eliminá-los quando necessário. O sistema valida os formatos e garante que apenas ficheiros válidos são processados.
    
    \item \textbf{Gestão de quarters:} cada utilizador pode criar períodos identificados como \textit{Quarter N}, para organizar os ficheiros carregados. Tem também de ser possível visualizar a lista de quarters disponíveis.

    \item \textbf{Autenticação:} o sistema deve permitir a criação de contas e o login por parte dos utilizadores, garantindo que cada um acede apenas aos seus próprios dados.
    
\end{itemize}

\subsubsection{Requisitos não funcionais}

Alguns requisitos não funcionais foram igualmente críticos para garantir a robustez e usabilidade do sistema. Os requisitos identificados foram os seguintes:

\begin{itemize}
    \item \textbf{Usabilidade:} a interface deve ser intuitiva, suportar múltiplos browsers e permitir o carregamento progressivo de gráficos sem bloquear a interação do utilizador (\textit{lazy load}).
    
    \item \textbf{Segurança:} o sistema deve garantir uma autenticação segura e assegurar que cada utilizador ou grupo apenas consegue aceder aos seus próprios dados. Deve também assegurar que só permite carregar ficheiros com o formato previsto.
    
    \item \textbf{Performance:} a aplicação deve ser capaz de suportar múltiplos utilizadores em simultâneo sem degradação significativa, mantendo uma resposta rápida às interações.
    
    \item \textbf{Acessibilidade:} o sistema deve ser acessível por teclado e compatível com leitores de ecrã (\textit{screen-reader friendly}).

    \item \textbf{Normalização de dados:} os dados extraídos dos ficheiros devem ser automaticamente normalizados para garantir consistência e compatibilidade com o sistema de visualização.

\end{itemize}

Os tabelas de requisitos por inteiro estão incluidas em apêndice (\cf, capítulo \ref{ch:tabRequisitos}).

\section{Casos de Utilização}

Com base nos requisitos funcionais e não funcionais identificados na secção anterior, foram definidos os casos de utilização que iremos suportar. O diagramas correspondente, tal como a matriz de prioridade, encontram-se no (\cf, capítulo \ref{ch:casosUtilizacao}), mas em resumo, os casos de utilização são os que se seguem:

\begin{enumerate}
    \item \textbf{Registo de utilizador: } \\
    O utilizador acede à plataforma pela primeira vez e opta por criar uma conta. O sistema solicita os dados de autenticação e valida se o utilizador já existe. Caso não exista, a conta é criada e o utilizador é autenticado automaticamente. \\
    \textit{Este caso responde ao requisito funcional de autenticação e está alinhado com os requisitos não funcionais de segurança.}
    
    \item \textbf{Autenticação do utilizador: } \\
    Ao entrar na aplicação, o utilizador introduz o seu username e password. O sistema valida o login com base nos dados guardados e, se for bem-sucedido, redireciona o utilizador para a página principal. \\
    \textit{Relaciona-se com os requisitos de segurança e experiência de utilizador.}
    
    \item \textbf{Criação de um quarter: } \\
    O utilizador, já autenticado, pode criar um novo quarter, identificando-o por um número. O sistema garante que esse número é único para o utilizador. \\
    \textit{Este caso reflete o requisito funcional de organização por quarters, e está diretamente ligado à estrutura de dados definida no modelo.}
    
    \item \textbf{Carregamento de ficheiros: } \\
    O utilizador escolhe um quarter e carrega ficheiros exportados da plataforma de simulação. O sistema valida a extensão, processa os ficheiros, aplica a pipeline de normalização para cada folha do Excel. Os dados serão associados ao quarter de onde foram criados. \\
    \textit{Este caso é central no sistema e concretiza os requisitos de Carregamento, normalização dos dados, e compatibilidade com os ficheiros da plataforma Marketplace Simulations.}

    \item \textbf{Eliminação de ficheiros: } \\
    O utilizador pode eliminar ficheiros previamente carregados. O sistema remove as referências no backend, marca os ficheiros antigos como inativos e evita que continuem a ser usados na geração dos gráficos. \\
    \textit{Este caso garante a gestão de dados pelo utilizador e reforça o princípio de manter apenas uma versão ativa por ficheiro.}
    
    \item \textbf{Visualização de gráficos: } \\
    Após carregar os ficheiros, o utilizador pode aceder a gráficos criados a partir dos dados normalizados. Os gráficos são interativos, e o utilizador pode aplicar filtros (por exemplo, selecionar o país ou o quarter atual) para ajustar a visualização. Apenas os ficheiros mais recentes são utilizados para construir os gráficos. \\
    \textit{Este caso corresponde ao requisito de visualizações interativas e integra a lógica de isolamento de dados, performance e usabilidade.}
    
\end{enumerate}

Com os casos de utilização estabelecidos, foram identificados então os requisitos da plataforma, que podemos separar em requisitos funcionais e não funcionais.

%%________________________________________________________________________
\section{Fundamentos}
\label{sec:fundamentos}

A Marketplace Simulations  é uma empresa que desenvolve plataformas de simulação para fins educativos, ou seja, ferramentas que colocam os estudantes numa espécie de "jogo" onde cada equipa gere a sua própria empresa e compete com os colegas em cenários de mercado realistas. Isto acaba por funcionar como uma espécie de laboratório virtual, usado em faculdades de gestão e economia, como no \gls{iscal}, onde os alunos aplicam os conceitos aprendidos numa experiência em contexto educativo.

No caso concreto do nosso projeto, o módulo em questão chama-se \textit{International Corporate Management} (referida doravante como plataforma de simulação), e é utilizado tipicamente no último semestre, na cadeira Projeto de Simulação em Negócios Internacionais  da Licenciatura de Comércio e Negócios Internacionais. Aqui, cada grupo de alunos representa uma empresa que tem de atuar num mercado competitivo — tomando decisões sobre posicionamento de produto, investimento, preços, distribuição, contratação de equipas e por aí fora. Essas decisões são processadas pela plataforma, que simula o comportamento do mercado com base num algoritmo interno. 

No final de cada ronda (ou trimestre), os dados são disponibilizados na plataforma nas várias secções disponiveis numa visualização tabular, o que, na prática, faz com que os alunos saltem entre secções, fazer gráficos à mão e tirar conclusões com base em tabelas.

Em termos de estrutura temporal, a simulação decorre ao longo de vários períodos, normalmente designados por quarters. Cada um destes quarters representa uma fase do ciclo de vida da empresa simulada — incluindo decisões estratégicas, execução e análise de resultados. Ou seja, no final de cada trimestre, os alunos recebem os dados com os resultados das decisões anteriores, o que obriga a uma análise comparativa constante entre períodos. É precisamente esta lógica iterativa — decidir, analisar, ajustar, repetir — que dá ritmo à simulação e aproxima o exercício de uma situação real de gestão empresarial.


%%________________________________________________________________________
\section{Abordagem}
\label{sec:abordagem}
%%________________________________________________________________________

O projeto proposto pretende então ajudar nesse sentido, de conseguir transformar dados desta plataforma em algo mais fácil e rápido de analisar. Tento o contexto da plataforma, iremos então descrever duas abordagens que considerámos.

\subsection{Web Scraping}
A primeira abordagem que consideramos foi a hipótese de automatizar a extração dos dados diretamente da plataforma do Marketplace Simulations, através de técnicas de web scraping. A ideia parecia interessante numa fase conceptual, já que permitiria reduzir a dependência do utilizador no processo de exportação manual dos dados. No entanto, rapidamente percebemos que esta abordagem trazia vários desafios que, na prática, a tornavam pouco viável, ou mesmo arriscada.

Primeiro, cada conta na plataforma está associada a um grupo de alunos, ou seja, é uma conta ativa e personalizada, usada diretamente durante a simulação. Isto significa que qualquer processo automático que iniciasse sessão, mesmo que fosse só para leitura, poderia inadvertidamente interagir com a interface e acabar por alterar alguma opção crítica — o que seria desastroso num contexto académico em que cada decisão tem impacto na avaliação. Além disso, como o acesso à plataforma é feito por licenças pagas, não existe qualquer possibilidade de criar uma service account ou utilizador apenas para leitura com permissões de administração. Ou seja, qualquer tentativa de scraping teria de reutilizar credenciais reais, o que levanta não só questões de segurança, mas também (possivelmente) legais.  

Outro fator que pesou na decisão foi o próprio risco técnico do scraping: plataformas deste tipo estão muitas vezes protegidas com mecanismos anti-automação (por exemplo: desafios CAPTCHA), e não conhecendo em detalhe a aplicação, poderíamos facilmente encontrar barreiras inesperadas, ou mesmo cair em práticas que fossem contra os termos de uso do serviço.

Por todos estes motivos, optámos por não seguir esta via. Em vez disso, definimos como parte do fluxo normal da aplicação que os próprios alunos devem exportar dados a partir da plataforma de simulação e, de seguida, carregar para a nossa aplicação. Esta solução, embora mais manual, garante segurança, respeita a integridade das contas dos utilizadores, e evita problemas legais ou técnicos com a aplicação de simulação.

\subsection{Exportação e carregamento manual de ficheiros}

A abordagem que acabamos por usar foi os alunos exportam manualmente os dados diretamente da plataforma de simulação e carregarem esses ficheiros na nossa plataforma. A partir daí, o sistema processa automaticamente os dados, normaliza os conteúdos e cria visualizações interativas. Esta abordagem, apesar de requerer uma ação manual inicial por parte do utilizador, é mais segura e prática quando comparada com a alternativa de web scraping, que, como vimos acima, apresentava vários desafios que poderiam ir contra os termos do serviço. Deste modo, garantimos um equilíbrio entre usabilidade, segurança e fiabilidade do sistema. Acabamos então por criar um \gla{sad} (\cf, capítulo \ref{ch:sad}) que permita os estudantes tomarem melhores decisões.


Para isso, procurámos que a nossa aplicação refletisse a estrutura da própria simulação. Como tal, os dados são organizados por quarters, como acontece na plataforma de simulação e permitir o carregamento dos dados exportados. Estes ficheiros são depois processados o que nos permite trabalhar com dados mais consistentes. 

Esta estrutura base implica a existência de três entidades principais na nossa aplicação (que iremos descrever a seguir) e cuja interação define o funcionamento base do sistema.


\subsubsection{\textit{Quarters}}
Os quarters funcionam como \textit{buckets} lógicos para organizar os ficheiros carregados pelos utilizadores. Cada utilizador pode criar múltiplos quarters, identificados de forma única por um número. Este número serve tambem de identificador, sendo que não é possível ter dois quarters identificados como 1 para cada utilizador.

A nível de implementação, cada quarter está associado unicamente a um utilizador e é identificado por um UUID (gerado automaticamente pelo \textit{Django}), que garante que o quarter seja único.

\subsubsection{Ficheiros}

Os ficheiros são inicialmente carregados no formato \gls{xlsx}, contendo uma ou várias folhas de cálculo. Cada folha é tratada como uma entidade individual e transformada para \gls{csv}. O ficheiro \gls{xlsx} é guardado como referência, mas não é diretamente utilizado para visualização.

Este processo de conversão para \gls{csv} é acompanhado por uma \textit{pipeline} de normalização de dados, que limpa os dados (como por exemplo, remover quebras de linha, colunas sem representação, nomes inconsistentes) e aplica regras para que os gráficos possam ser gerados de forma consistente. Cada ficheiro \gls{csv} gerado é associado ao seu ficheiro \gls{xlsx} de origem, ao quarter correspondente, e o seu nome será baseado no nome da folha de onde foi extraído. Aos ficheiros \gls{csv} é também associado uma \textit{slug}, baseado também no nome, que identifica a informação que o \gls{csv} representa.

A plataforma garante que só existe uma versão ativa de cada ficheiro por tipo — caso o utilizador carregue novamente um ficheiro com o mesmo nome lógico, o anterior será marcado como não ativo, evitando duplicações e garantindo que os gráficos usam apenas dados mais recentes.

\subsubsection{Utilizadores}

A plataforma foi desenhada para funcionar com utilizadores. Cada utilizador tem a sua conta, e pode criar quarters, carregar ou alterar ficheiros, e ver aos gráficos gerados a partir desses ficheiros.

Cada utilizador tem acesso apenas aos seus próprios dados, garantindo o isolamento da informação. Esta separação é feita a nível da base de dados, através da associação de cada entidade ao utilizador que criou.

Apesar da plataforma não suportar explicitamente equipas ou grupos, assume-se que alunos do mesmo grupo podem carregar ficheiros semelhantes, mas o sistema trata-os como ficheiros diferentes. Assim, evita-se a complexidade adicional de gerir permissões ou partilha de dados entre contas. Também se assume que as contas podem ser criadas ao nível do grupo, pelo que para a plataforma, é indiferente se a conta é individual ou partilhada entre membros desse grupo.

No futuro, pode ser considerada a funcionalidade de desativação automática de contas (por exemplo, após o final do semestre), mas para já o modelo é simples e robusto: conta individual, dados isolados, e controlo completo sobre os próprios uploads.

\subsubsection{\textit{Pipeline} de Normalização de Dados}

Outro contributo diferenciador foi o desenvolvimento de uma \textit{pipeline} modular de normalização de dados. O objetivo é garantir que os ficheiros carregados, que muitas vezes contêm nomes de colunas inconsistentes, quebras de linha, espaços em excesso ou colunas irrelevantes, sejam adaptados para serem visualizados. 

Como podemos receber muitos ficheiros, a variabilidade entre os dados recebidos é muito alta, pelo que alguns dados passam por mais do que uma fase de normalização. Esta decisão foi tomada com base numa análise manual, em que identificámos possíveis fontes de dados que precisam de mais do que uma fase de normalização. As várias fases de normalização alteram os dados de modo a facilitar a representação visual dos mesmos e é um passo essencial no projeto, porque garante que a aplicação trabalha com formatos e regras conhecidas, e remove a variabilidade dos ficheiros importados.

As fases de normalização irão ser descritas em mais detalhe nos capítulos seguintes, uma vez que a implementação destas pipeline estão relacionadas à tecnologia escolhida, mas o desenvolvimento desta \textit{pipeline} é um fator diferenciador deste projeto, uma vez que tem de lidar com dados que não estão estruturados de forma a facilitar representações visuais. 

Este processo de normalização é semelhante aos processos \gls{etl} ainda que neste projeto tenha sido desenvolvido com uma escala menor.

