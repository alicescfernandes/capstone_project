%%________________________________________________________________________
%% LEIM | PROJETO
%% 2022 / 2013 / 2012
%% Modelo para relatório
%% v04: alteração ADEETC para DEETC; outros ajustes
%% v03: correção de gralhas
%% v02: inclui anexo sobre utilização sistema controlo de versões
%% v01: original
%% PTS / MAR.2022 / MAI.2013 / 23.MAI.2012 (construído)
%%________________________________________________________________________




%%________________________________________________________________________
\myPrefaceChapter{Resumo}
%%________________________________________________________________________

Este  relatório apresenta o desenvolvimento de uma ferramenta de visualização de dados, projetada para apoiar os alunos que utilizam a plataforma \textit{Marketplace Simulations - International Corporate Management}, na cadeira Projeto de Simulação em Negócios Internacionais da Licenciatura de Comércio e Negócios Internacionais \cite{FUC_ISCAL_2025} durante ultimo semestre do curso . O principal objetivo da aplicação é simplificar a análise de dados, permitindo aos alunos carregar ficheiros exportados da plataforma de simulação, processar e normalizar a informação carregada e mostrar gráficos com base nessa informação.

A solução proposta introduz um modelo de dados baseado nos períodos de simulação da plataforma, em que os utilizadores podem gerir os dados. Cada ficheiro carregado é analisado e transformado em dados estruturados através de uma \textit{pipeline} de normalização, desenvolvida utilizando \textit{Python} e \textit{Pandas}. Os conjuntos de dados resultantes são visualizados através de gráficos com recurso ao \textit{Plotly} incorporados na interface gráfica \textit{web}.

O projeto utiliza \textit{Django} como a biblioteca base. As principais contribuições incluem o carregamento de ficheiros, automatismos no processamento de dados e uma interface gráfica que vai de encontro às necessidades dos utilizadores. O sistema foi desenvolvido de forma incremental e está preparado para ser implementado num ambiente de produção, suportando vários utilizadores e com controlo de versões.


%%________________________________________________________________________
\myPrefaceChapter{Abstract}
%%________________________________________________________________________

This report presents the development of a data visualization tool designed to support students using the \textit{Marketplace Simulations - International Corporate Management} platform during the class International Business Simulation Project \cite{FUC_ISCAL_2025} of the Bachelor's Degree in International Trade and Business of ISCAL, during the final semester of the program. The main objective of the application is to simplify data analysis by allowing students to upload files exported from the simulation platform, process and normalize the uploaded information, and display charts based on that information.

The proposed solution introduces a data model based on the simulation periods of the platform, where users can manage the data. Each uploaded file is analyzed and transformed into structured data through a normalization \textit{pipeline}, developed using \textit{Python} and \textit{Pandas}. The resulting datasets are visualized through charts using \textit{Plotly}, embedded within the graphical \textit{web} interface.

The project uses \textit{Django} as the core framework. The main contributions include file uploads, automated data processing, and a graphical interface tailored to the users’ needs. The system was developed incrementally and is ready to be deployed in a production environment, supporting multiple users and version control.