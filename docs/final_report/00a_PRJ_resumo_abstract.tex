%%________________________________________________________________________
%% LEIM | PROJETO
%% 2022 / 2013 / 2012
%% Modelo para relatório
%% v04: alteração ADEETC para DEETC; outros ajustes
%% v03: correção de gralhas
%% v02: inclui anexo sobre utilização sistema controlo de versões
%% v01: original
%% PTS / MAR.2022 / MAI.2013 / 23.MAI.2012 (construído)
%%________________________________________________________________________




%%________________________________________________________________________
\myPrefaceChapter{Resumo}
%%________________________________________________________________________

Neste trabalho apresentamos o desenvolvimento de uma aplicação \textit{web} para visualização de dados, pensada para facilitar a análise da informação exportada da plataforma \textit{Marketplace Simulations – International Corporate Management}, usada na unidade curricular Projeto de Simulação em Negócios Internacionais, lecionada no último semestre da Licenciatura em Comércio e Negócios Internacionais do ISCAL.

A motivação principal surgiu da ausência de visualizações integradas ou ferramentas de apoio à análise na plataforma, que apesar disso permite exportar vários ficheiros Excel com dados relevantes. Assim, o objetivo da aplicação passa por permitir aos alunos carregar os ficheiros exportados, tratar e normalizar automaticamente a informação, e visualizar os resultados através de gráficos interativos, reduzindo o esforço manual e aumentando a clareza dos dados.

A solução foi organizada com base nos períodos de simulação da plataforma, designados por \textit{quarters}, permitindo ao utilizador agrupar e consultar os dados por fase. Cada ficheiro carregado passa por uma \textit{pipeline} de normalização implementada em \textit{Python}, com recurso à biblioteca \textit{Pandas}, que trata da normalização e estruturação dos dados. A visualização final é feita com \textit{Plotly}, integrada diretamente na interface gráfica da aplicação.

A aplicação foi desenvolvida com \textit{Django}, onde foi implementada toda a lógica de backend, autenticação de utilizadores, gestão de \textit{quarters} e ficheiros, bem como os endpoints necessários para alimentar os gráficos. A interface visual utiliza as bibliotecas \textit{Flowbite}, \textit{Plotly} e \textit{Datatables}, garantindo uma navegação simples e consistente. A aplicação foi construída de forma incremental e modular, e encontra-se preparada para múltiplos utilizadores com isolamento de dados e controlo de versões.

Como resultado, obteve-se uma ferramenta que facilitou a exploração visual dos dados simulados. A solução contribuiu para uma análise mais acessível e interativa do seu desempenho (anda que seja uma simulação), apoiando a tomada de decisões fundamentadas no contexto académico. Para além disso, promoveu a autonomia dos utilizadores na interpretação dos dados, reforçando o valor pedagógico da simulação.

%%________________________________________________________________________
\myPrefaceChapter{Abstract}
%%________________________________________________________________________

In this paper, we present the development of a web application for data visualization, designed to facilitate the analysis of information exported from the Marketplace Simulations's International Corporate Management platform, used in the International Business Simulation Project course, taught in the last semester of the Bachelor's Degree in International Trade and Business at ISCAL.

The main motivation arose from the absence of integrated visualizations or analysis support tools on the platform, which nevertheless allows the export of several Excel files with relevant data. Thus, the objective of the application is to allow students to upload the exported files, automatically process and standardize the information, and visualize the results through interactive graphs, reducing manual effort and increasing data clarity.

The solution was organized based on the platform's simulation periods, called quarters, allowing the user to group and consult the data by phase. Each uploaded file goes through a normalization pipeline implemented in Python, using the Pandas library, which handles data normalization and structuring. The final visualization is done with Plotly, integrated directly into the application's graphical interface.

The application was developed with Django, where all the backend logic, user authentication, quarters and file management, as well as the endpoints needed to feed the graphs, were implemented. The interface use the frameworks \textit{Flowbite}, \textit{Plotly} and \textit{Datatables}, ensuring simple and consistent navigation. The application was built incrementally, and is prepared for multiple users with data isolation and version control.

As a result, a tool was obtained that facilitated the visual exploration of the simulated data. The solution contributed to a more accessible and interactive analysis of its performance (even though it is a simulation), supporting informed decision-making in the academic context. In addition, it promoted user autonomy in interpreting the data, reinforcing the pedagogical value of the simulation.
