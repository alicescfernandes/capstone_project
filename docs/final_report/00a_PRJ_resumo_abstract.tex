%%________________________________________________________________________
%% LEIM | PROJETO
%% 2022 / 2013 / 2012
%% Modelo para relatório
%% v04: alteração ADEETC para DEETC; outros ajustes
%% v03: correção de gralhas
%% v02: inclui anexo sobre utilização sistema controlo de versões
%% v01: original
%% PTS / MAR.2022 / MAI.2013 / 23.MAI.2012 (construído)
%%________________________________________________________________________




%%________________________________________________________________________
\myPrefaceChapter{Resumo}
%%________________________________________________________________________

Neste trabalho apresentamos o desenvolvimento de uma aplicação web para visualização de dados, pensada para facilitar a análise da informação exportada da plataforma \textit{International Corporate Management} da empresa \textit{Marketplace Simulations}, usada na unidade curricular Projeto de Simulação em Negócios Internacionais, lecionada na Licenciatura em Comércio e Negócios Internacionais do \gls{iscal}.

A motivação para este projeto surgiu da ausência de visualizações integradas na plataforma, o que dificulta a leitura dos dados e obriga os alunos a analisarem manualmente vários ficheiros Excel exportados. A nossa solução visa simplificar este processo, permitindo carregar os ficheiros diretamente na aplicação e gerar automaticamente gráficos interativos com filtros avançados, reduzindo o esforço manual e melhorando a sua interpretação.

A aplicação foi desenvolvida com \textit{Django} e organiza os dados por períodos de simulação, designados por \textit{quarters}, garantindo acessos isolado por utilizador. Cada ficheiro carregado é processado por um \textit{pipeline} de normalização de dados, implementado com \textit{Pandas}, que transforma a informação extraída dos ficheiros carregados em dados consistentes e preparados para serem mostrados. Esta transformação inclui limpeza automática de colunas irrelevantes, correção de cabeçalhos e uniformização do conteúdo, entre outras transformações feitas.

A interface gráfica foi construída com recurso às bibliotecas \textit{Flowbite}, \textit{Plotly} e \textit{Datatables}, e oferece uma experiência intuitiva. Os gráficos são gerados dinamicamente com base nos dados processados, e suportam navegação entre \textit{quarters}, seleção de parâmetros e filtros. A arquitetura do sistema permite escalabilidade, reutilização de componentes e gestão autónoma por parte dos utilizadores.

Como resultado, a aplicação contribui para uma análise mais acessível e interativa dos dados da simulação, apoiando a tomada de decisões.

%%________________________________________________________________________
\myPrefaceChapter{Abstract}
%%________________________________________________________________________

In this paper, we present the development of a web application for data visualization, designed to facilitate the analysis of information exported from the \textit{International Corporate Management} platform of \textit{Marketplace Simulations}, used in the \textit{International Business Simulation} course, taught in the Bachelor's Degree in International Trade and Business at \gls{iscal}.

The motivation for this project arose from the lack of integrated visualizations in the platform, which makes it difficult to read the data and forces students to manually analyze several exported Excel files. Our solution aims to simplify this process by allowing files to be uploaded directly to the application and automatically generating interactive graphs with advanced filters, reducing manual effort and improving interpretation.

The application was developed with \textit{Django} and organizes data by simulation periods, called quarters, ensuring isolated access per user. Each uploaded file is processed by a data normalization pipeline, implemented with \textit{Pandas}, which transforms the information extracted from the uploaded files into consistent data ready to be displayed. This transformation includes automatic cleaning of irrelevant columns, correction of headers, and standardization of content, among other transformations.

The graphical interface was built using the \textit{Flowbite}, \textit{Plotly}, and \textit{Datatables} libraries and offers an intuitive experience. The graphs are dynamically generated based on the processed data and support navigation between quarters, parameter selection, and filters. The system architecture allows for scalability, component reuse, and autonomous management by users.

As a result, the application contributes to a more accessible and interactive analysis of simulation data, supporting decision-making.