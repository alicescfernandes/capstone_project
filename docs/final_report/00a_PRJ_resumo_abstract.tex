%%________________________________________________________________________
%% LEIM | PROJETO
%% 2022 / 2013 / 2012
%% Modelo para relatório
%% v04: alteração ADEETC para DEETC; outros ajustes
%% v03: correção de gralhas
%% v02: inclui anexo sobre utilização sistema controlo de versões
%% v01: original
%% PTS / MAR.2022 / MAI.2013 / 23.MAI.2012 (construído)
%%________________________________________________________________________




%%________________________________________________________________________
\myPrefaceChapter{Resumo}
%%________________________________________________________________________

Neste trabalho apresentamos o desenvolvimento de uma aplicação \textit{web} para visualização e análise de dados, criada com o objetivo de apoiar os alunos na interpretação da informação extraída da plataforma \textit{International Corporate Management}, utilizada na unidade curricular Projeto de Simulação em Negócios Internacionais da Licenciatura em Comércio e Negócios Internacionais do \gls{iscal}.

A plataforma permite extrair a informação apresentada aos alunos ao longo da simulação, organizada por diferentes áreas funcionais. No entanto, os dados extraídos apresentam uma grande diversidade estrutural, com problemas como nomes de colunas com quebras de linha ou com muitos espaços, presença de colunas irrelevantes e valores mal formatados. Como estes dados servem de base para visualizações e análises interativas, tornou-se necessário garantir que estivessem num formato tratado, coerente e estruturado. Para isso, foi desenvolvido uma \textit{pipeline} de normalização recorrendo a \textit{Python}, com operadores responsáveis por tarefas como a extração e normalização dos cabeçalhos, remoção de colunas desnecessárias, limpeza de conteúdo das células e transformação de valores nomeadamente no caso de métricas financeiras. Os nomes dos ficheiros normalizados são também criados com base na folha de origem, garantindo consistência. Este processo assegura que a informação fica preparada para ser visualizada forma coerente ao longo dos diferentes períodos da simulação.

A aplicação foi construída sobre o framework \textit{Django}, que assegura a gestão de contas de utilizador, a organização dos dados por períodos (quarters) e a separação da informação por utilizador. Esta arquitetura garante que cada conta tem acesso apenas aos seus próprios dados, promovendo o isolamento e a integridade da informação. A interface gráfica combina \textit{Flowbite}, \textit{Plotly} e \textit{WebComponents}, permitindo uma experiência de utilização fluida, intuitiva e compatível com diferentes dispositivos.

Do ponto de vista funcional, a aplicação permite criar e gerir períodos de simulação, carregar conjuntos de dados, extrair automaticamente a informação relevante, aplicar filtros por parâmetros como marca, cidade ou necessidade do cliente, e visualizar gráficos configurados com base na estrutura dos dados reconhecidos.

A solução foi desenhada com foco na modularidade do \textit{pipeline} de transformação e na experiência de utilização, com o objetivo de apoiar a tomada de decisão por parte dos alunos durante a simulação.

%%________________________________________________________________________
\myPrefaceChapter{Abstract}
%%________________________________________________________________________

In this paper, we present the development of a web application for data visualization and analysis, created with the aim of supporting students in interpreting information extracted from the International Corporate Management platform, used in the International Business Simulation Project course of the Bachelor's Degree in International Trade and Business at \gls{iscal}.

The platform allows the extraction of information presented to students throughout the simulation, organized by different functional areas. However, the extracted data presents a great structural diversity, with problems such as column names with line breaks or too many spaces, the presence of irrelevant columns, and poorly formatted values. As this data serves as the basis for interactive visualizations and analyses, it became necessary to ensure that it was in a treated, coherent, and structured format. To this end, a normalization pipeline was developed using Python, with operators responsible for tasks such as extracting and normalizing headers, removing unnecessary columns, cleaning cell content, and transforming values, particularly in the case of financial metrics. The names of the normalized files are also created based on the source sheet, ensuring consistency. This process ensures that the information is prepared to be viewed consistently throughout the different periods of the simulation.

The application was built on the Django framework, which ensures user account management, data organization by periods (quarters), and separation of information by user. This architecture ensures that each account has access only to its own data, promoting isolation and information integrity. The graphical interface combines Flowbite, Plotly, and WebComponents, allowing for a fluid, intuitive user experience that is compatible with different devices.

From a functional point of view, the application allows you to create and manage simulation periods, load data sets, automatically extract relevant information, apply filters by parameters such as brand, city, or customer need, and view graphs configured based on the structure of the recognized data.

The solution was designed with a focus on the modularity of the transformation pipeline and the user experience, with the aim of supporting students' decision-making during the simulation.
