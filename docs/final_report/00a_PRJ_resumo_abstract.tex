%%________________________________________________________________________
%% LEIM | PROJETO
%% 2022 / 2013 / 2012
%% Modelo para relatório
%% v04: alteração ADEETC para DEETC; outros ajustes
%% v03: correção de gralhas
%% v02: inclui anexo sobre utilização sistema controlo de versões
%% v01: original
%% PTS / MAR.2022 / MAI.2013 / 23.MAI.2012 (construído)
%%________________________________________________________________________




%%________________________________________________________________________
\myPrefaceChapter{Resumo}
%%________________________________________________________________________

Neste trabalho apresentamos o desenvolvimento de uma aplicação \textit{web} para visualização de dados, pensada para facilitar a análise da informação exportada da plataforma \textit{International Corporate Management} da empresa \textit{Marketplace Simulations}, usada na unidade curricular Projeto de Simulação em Negócios Internacionais, lecionada na Licenciatura em Comércio e Negócios Internacionais do ISCAL.

A motivação principal surgiu da ausência de visualizações integradas na plataforma, que dificulta a leitura dos dados, mas que permite exportar vários ficheiros \textit{Excel} com a informação necessária. Assim, o objetivo da nossa aplicação passa por permitir aos alunos carregar os ficheiros exportados, e visualizar os resultados através de gráficos, reduzindo o esforço manual e aumentando a clareza dos dados.

A solução foi organizada com base nos períodos de simulação da plataforma de simulação, permitindo ao utilizador agrupar e consultar os dados por fase. Cada ficheiro carregado passa por uma \textit{pipeline} de tratamento de dados implementada em \textit{Python} utilizando a biblioteca \textit{Pandas}. Os gráficos são feitos com recurso à biblioteca \textit{Plotly}, integrada na interface gráfica da aplicação.

A aplicação foi desenvolvida com \textit{Django}, onde foi implementada toda a lógica de servidor. A interface visual utiliza as bibliotecas \textit{Flowbite}, \textit{Plotly} e \textit{Datatables}, que garantem uma interface intuitiva e responsiva. A aplicação foi construída de forma incremental e modular, e encontra-se preparada para múltiplos utilizadores.

Como resultado, a aplicação desenvolvida facilita a leitura dos dados. A solução contribuiu para uma análise mais acessível e interativa, apoiando a tomada de decisões em contexto académico.

%%________________________________________________________________________
\myPrefaceChapter{Abstract}
%%________________________________________________________________________

In this paper, we present the development of a web application for data visualization, designed to facilitate the analysis of information exported from the International Corporate Management platform of Marketplace Simulations, used in the International Business Simulation Project course, taught in the Bachelor's Degree in International Trade and Business at ISCAL.

The main motivation arose from the absence of integrated visualizations in the platform, which makes it difficult to read the data, but allows several Excel files with the necessary information to be exported. Thus, the goal of our application is to allow students to upload the exported files and view the results through graphs, reducing manual effort and increasing data clarity.

The solution was organized based on the simulation periods of the simulation platform, allowing the user to group and consult the data by phase. Each uploaded file goes through a data processing pipeline implemented in Python using the Pandas library. The graphs are made using the Plotly library, integrated into the application's graphical interface.

The application was developed with Django, where all server logic was implemented. The visual interface uses the Flowbite, Plotly, and Datatables libraries, which ensure an intuitive and responsive interface. The application was built incrementally and modularly and is ready for multiple users.

As a result, the developed application facilitates data reading. The solution contributed to a more accessible and interactive analysis, supporting decision-making in an academic context.