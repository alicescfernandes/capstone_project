%%________________________________________________________________________
%% LEIM | PROJETO
%% 2022 / 2013 / 2012
%% Modelo para relatório
%% v04: alteração ADEETC para DEETC; outros ajustes
%% v03: correção de gralhas
%% v02: inclui anexo sobre utilização sistema controlo de versões
%% v01: original
%% PTS / MAR.2022 / MAI.2013 / 23.MAI.2012 (construído)
%%________________________________________________________________________




%%________________________________________________________________________
\myPrefaceChapter{Resumo}
%%________________________________________________________________________

Neste trabalho apresentamos o desenvolvimento de uma aplicação \textit{web} para visualização de dados, pensada para facilitar a análise da informação exportada da plataforma \textit{Marketplace Simulations – International Corporate Management}, usada na unidade curricular Projeto de Simulação em Negócios Internacionais, lecionada no último semestre da Licenciatura em Comércio e Negócios Internacionais do ISCAL.

A motivação principal surgiu da ausência de visualizações integradas ou ferramentas de apoio à análise na plataforma, que apesar disso permite exportar vários ficheiros Excel com dados relevantes. Assim, o objetivo da aplicação passa por permitir aos alunos carregar os ficheiros exportados, tratar e normalizar automaticamente a informação, e visualizar os resultados através de gráficos interativos, reduzindo o esforço manual e aumentando a clareza dos dados.

A solução foi organizada com base nos períodos de simulação da plataforma, que aqui designamos por \textit{quarters}, permitindo ao utilizador agrupar e consultar os dados por fase. Cada ficheiro carregado passa por uma \textit{pipeline} de normalização implementada em \textit{Python}, com recurso à biblioteca \textit{Pandas}, que trata da normalização e estruturação dos dados. A visualização final é feita com \textit{Plotly}, integrada diretamente na interface gráfica da aplicação.

O sistema foi desenvolvido com \textit{Django}, onde foi implementada toda a lógica de backend, autenticação de utilizadores, gestão de \textit{quarters} e ficheiros, bem como os endpoints necessários para alimentar os gráficos. A interface segue o sistema de design \textit{Flowbite}, garantindo uma navegação simples e consistente. A aplicação foi construída de forma incremental e modular, e encontra-se preparada para múltiplos utilizadores com isolamento de dados e controlo de versões.

%%________________________________________________________________________
\myPrefaceChapter{Abstract}
%%________________________________________________________________________

In this project, we present the development of a \textit{web} application for data visualization, designed to simplify the analysis of information exported from the \textit{Marketplace Simulations – International Corporate Management} platform, used in the course "International Business Simulation Project", taught in the final semester of the Bachelor's degree in International Trade and Business at ISCAL.

The main motivation came from the lack of built-in visualizations or analysis tools in the platform, which, despite this, allows users to export several Excel files containing relevant data. Therefore, the goal of the application is to enable students to upload the exported files, automatically clean and normalize the data, and visualize the results through interactive charts — reducing manual effort and improving clarity.

The solution is structured around the simulation periods defined by the platform, referred to here as \textit{quarters}, allowing users to group and explore data by phase. Each uploaded file is processed by a normalization \textit{pipeline} implemented in \textit{Python}, using the \textit{Pandas} library to standardize and structure the data. The final visualizations are generated with \textit{Plotly} and integrated directly into the \textit{web} interface.

The system was developed using the \textit{Django} framework, which handles all backend logic, user authentication, management of \textit{quarters} and uploaded files, as well as the API endpoints that feed the charts. The interface follows the \textit{Flowbite} design system, ensuring a clean and user-friendly experience. The application was built incrementally with a modular approach and is ready to support multiple users, with proper data isolation and version control mechanisms.