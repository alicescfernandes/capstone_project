%%________________________________________________________________________
%% LEIM | PROJETO
%% 2022 / 2013 / 2012
%% Modelo para relatório
%% v04: alteração ADEETC para DEETC; outros ajustes
%% v03: correção de gralhas
%% v02: inclui anexo sobre utilização sistema controlo de versões
%% v01: original
%% PTS / MAR.2022 / MAI.2013 / 23.MAI.2012 (construído)
%%________________________________________________________________________




%%________________________________________________________________________
\myPrefaceChapter{Resumo}
%%________________________________________________________________________

Este  relatório apresenta o desenvolvimento de uma ferramenta de visualização de dados, desenhada para apoiar os alunos que utilizam a  plataforma \textit{Marketplace Simulations - International Corporate Management}, durante o semestre. O principal objetivo da aplicação é simplificar a análise de dados, permitindo aos alunos carregar ficheiros Excel exportados da plataforma de simulação, processar e normalizar a informação carregada e mostrar gráficos interativos e dinâmicos.

A solução proposta introduz um modelo de dados baseado nos períodos de simulação da plataforma, em que os utilizadores podem gerir os dados. Cada ficheiro carregado é analisado e convertido em ficheiros CSV estruturados através de uma pipeline de normalização, construída utilizando \textit{Python} e \textit{Pandas}. Os conjuntos de dados resultantes são visualizados através de gráficos dinâmicos com recurso ao Plotly incorporados na interface gráfica \textit{web}.

O projeto segue um design modular e centrado no utilizador, utilizando o \textit{Django} como a biblioteca base da plataforma. As principais contribuições incluem a criação de um modelo de organização de ficheiros, automatismos no processamento de dados e uma interface gráfica intuitiva que vai de encontro às necessidades dos utilizadores. O sistema foi desenvolvido de forma incremental e está preparado para ser implementado num ambiente de produção, suportando vários utilizadores com isolamento de dados, controlo de versões.



%%________________________________________________________________________
\myPrefaceChapter{Abstract}
%%________________________________________________________________________

This report presents the development of a data visualization tool designed to support students using the Marketplace Simulations - International Corporate Management platform during the semester. The main aim of the application is to simplify data analysis by allowing users to upload Excel files exported from the simulation platform, automatically process and normalize the data and create interactive and dynamic graphs.

The proposed solution introduces a data model based on the platform's simulation periods, in which users can manage the data. Each uploaded file is analysed and converted into structured CSV files through a normalization pipeline built using \textit{Python} and \textit{Pandas}. The resulting data sets are visualized through dynamic graphs using Plotly incorporated into the \textit{web} graphical interface.

The project follows a modular, user-centered design, using \textit{Django} as the platform's base library. The main contributions include the creation of a file organization model, automated data processing and an intuitive graphical interface that meets users' needs. The system was developed incrementally and is ready to be implemented in a production environment, supporting multiple users with data isolation and version control.