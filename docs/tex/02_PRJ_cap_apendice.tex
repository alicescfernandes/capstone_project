%%________________________________________________________________________
%% LEIM | PROJETO
%% 2022 / 2013 / 2012
%% Modelo para relatório
%% v04: alteração ADEETC para DEETC; outros ajustes
%% v03: correção de gralhas
%% v02: inclui anexo sobre utilização sistema controlo de versões
%% v01: original
%% PTS / MAR.2022 / MAI.2013 / 23.MAI.2012 (construído)
%%________________________________________________________________________




%%________________________________________________________________________
\chapter{Tutorial}
\label{ch:umDetalheAdicional}
%%________________________________________________________________________

O \aspas{apêndice} utiliza-se para descrever aspectos que tendo sido desenvolvidos pelo autor constituem um complemento ao que já foi apresentado no corpo principal do documento.

Neste documento utilize o apêndice para explicar o processo usado na \textbf{gestão das versões} que foram sendo construídas ao longo do desenvolvimento do trabalho.

É especialmente importante explicar o objetivo de cada ramo (\aspas{branch}) definido no projeto (ou apenas dos ramos mais importantes) e indicar quais os ramos que participaram numa junção (\aspas{merge}).

É também importante explicar qual a arquitetura usada para interligar os vários repositórios (\eg, Git, GitHub, DropBox, GoogleDrive) que contêm as várias versões (e respetivos ramos) do projeto.


Notar a diferença essencial entre \aspas{apêndice} e \aspas{anexo}. O \aspas{apêndice} é um texto (ou documento) que descreve trabalho desenvolvido pelo autor (\eg, do relatório, monografia, tese). O \aspas{anexo} é um texto (ou documento) sobre trabalho que não foi desenvolvido pelo autor.

Para simplificar vamos apenas considerar a noção de \aspas{apêndice}. No entanto, pode sempre adicionar os anexos que entender como adequados.






%%________________________________________________________________________
\chapter{Casos de Utilização - UML}
\label{ch:casosUtilizacaoUml}
%%________________________________________________________________________

\chapter{Tabela de Requisitos funcionais}
\label{ch:outroDetalheAdicional}


\chapter{Tabela de Requisitos não funcionais}
\label{ch:outroDetalheAdicional}


\chapter{Classificação das folhas Excel}
\label{ch:outroDetalheAdicional}









